%\documentclass[defaultstyle,11pt]{thesis}
%\documentclass[]{report}
%\documentclass[]{article}
%\usepackage{aastex_hack}
%\usepackage{deluxetable}
\documentclass[preprint]{aastex}
%\documentclass{emulateapj}


%%%%%%%%%%%%%%%%%%%%%%%%%%%%%%%%%%%%%%%%%%%%%%%%%%%%%%%%%%%%%%%%
%%%%%%%%%%%  see documentation for information about  %%%%%%%%%%
%%%%%%%%%%%  the options (11pt, defaultstyle, etc.)   %%%%%%%%%%
%%%%%%%  http://www.colorado.edu/its/docs/latex/thesis/  %%%%%%%
%%%%%%%%%%%%%%%%%%%%%%%%%%%%%%%%%%%%%%%%%%%%%%%%%%%%%%%%%%%%%%%%
%		\documentclass[typewriterstyle]{thesis}
% 		\documentclass[modernstyle]{thesis}
% 		\documentclass[modernstyle,11pt]{thesis}
%	 	\documentclass[modernstyle,12pt]{thesis}

%%%%%%%%%%%%%%%%%%%%%%%%%%%%%%%%%%%%%%%%%%%%%%%%%%%%%%%%%%%%%%%%
%%%%%%%%%%%    load any packages which are needed    %%%%%%%%%%%
%%%%%%%%%%%%%%%%%%%%%%%%%%%%%%%%%%%%%%%%%%%%%%%%%%%%%%%%%%%%%%%%
\usepackage{latexsym}		% to get LASY symbols
\usepackage{graphicx}		% to insert PostScript figures
%\usepackage{deluxetable}
%\usepackage{rotating}		% for sideways tables/figures
\usepackage{natbib}  % Requires natbib.sty, available from http://ads.harvard.edu/pubs/bibtex/astronat/
\usepackage{savesym}
\usepackage{amssymb}
%\savesymbol{singlespace}
\savesymbol{doublespace}
%\usepackage{wrapfig}
%\usepackage{setspace}
\usepackage{xspace}
\usepackage{color}
%\usepackage{multicol}
\usepackage{mdframed}
\usepackage{url}
\usepackage{subfigure}
%\usepackage{emulateapj}
%\usepackage{lscape}
\usepackage{grffile}
\usepackage{standalone}
\standalonetrue
\usepackage{import}
\usepackage[utf8]{inputenc}
%\usepackage{longtable}
\usepackage{booktabs}



\newcommand{\rhoPDF}{\ensuremath{\rho-\mathrm{PDF}}\xspace}
\newcommand{\meanrho}{\ensuremath{\langle\rho\rangle}\xspace}
\newcommand{\GRSMC}{GRSMC 43.30-0.33\xspace}
\newcommand{\north}{G43.17+0.01\xspace}
\newcommand{\south}{G43.16-0.03\xspace}
\newcommand{\paa}{Pa\ensuremath{\alpha}}
\newcommand{\brg}{Br\ensuremath{\gamma}}
\newcommand{\msun}{\ensuremath{M_{\odot}}\xspace}			%  Msun
\newcommand{\mdot}{\ensuremath{\dot{M}}\xspace}
\newcommand{\lsun}{\ensuremath{L_{\odot}}}			%  Lsun
\newcommand{\lbol}{\ensuremath{L_{\mathrm{bol}}}}	%  Lbol
\newcommand{\ks}{K\ensuremath{_{\mathrm{s}}}}		%  Ks
\newcommand{\hh}{\ensuremath{\textrm{H}_{2}}\xspace}			%  H2
\newcommand{\formaldehyde}{\ensuremath{\textrm{H}_2\textrm{CO}}\xspace}
\newcommand{\formaldehydeIso}{\ensuremath{\textrm{H}_2~^{13}\textrm{CO}}\xspace}
\newcommand{\methanol}{\ensuremath{\textrm{CH}_3\textrm{OH}}\xspace}
\newcommand{\ortho}{\ensuremath{\textrm{o-H}_2\textrm{CO}}\xspace}
\newcommand{\oneone}{\ensuremath{1_{10}-1_{11}}\xspace}
\newcommand{\twotwo}{\ensuremath{2_{11}-2_{12}}\xspace}
\newcommand{\threethree}{\ensuremath{3_{12}-3_{13}}\xspace}
\newcommand{\threeohthree}{\ensuremath{3_{03}-2_{02}}\xspace}
\newcommand{\threetwotwo}{\ensuremath{3_{22}-2_{21}}\xspace}
\newcommand{\threetwoone}{\ensuremath{3_{21}-2_{20}}\xspace}
\newcommand{\JKaKc}{\ensuremath{J_{K_a K_c}}}
\newcommand{\water}{H$_{2}$O}		%  H2O
\newcommand{\feii}{\ion{Fe}{2}}		%  FeII
\newcommand{\uchii}{UC\ion{H}{2}\xspace}
\newcommand{\UCHII}{UC\ion{H}{2}\xspace}
\newcommand{\hii}{H~{\sc ii}\xspace}
\newcommand{\Hii}{H~{\sc ii}\xspace}
\newcommand{\HII}{H~{\sc ii}\xspace}
\newcommand{\kms}{\textrm{km~s}\ensuremath{^{-1}}\xspace}	%  km s-1
\newcommand{\nsample}{456\xspace}
\newcommand{\CFR}{5\xspace} % nMPC / 0.25 / 2 (6 for W51 once, 8 for W51 twice) REFEDIT: With f_observed=0.3, becomes 3/2./0.3 = 5
\newcommand{\permyr}{\ensuremath{\mathrm{Myr}^{-1}}\xspace}
\newcommand{\tsuplim}{0.5\xspace} % upper limit on starless timescale
\newcommand{\ncandidates}{18\xspace}
\newcommand{\mindist}{8.7\xspace}
\newcommand{\rcluster}{2.5\xspace}
\newcommand{\ncomplete}{13\xspace}
\newcommand{\middistcut}{13.0\xspace}
\newcommand{\nMPC}{3\xspace} % only count W51 once.  W51, W49, G010
\newcommand{\obsfrac}{30}
\newcommand{\nMPCtot}{10\xspace} % = nmpc / obsfrac
\newcommand{\nMPCtoterr}{6\xspace} % = sqrt(nmpc) / obsfrac
\newcommand{\plaw}{2.1\xspace}
\newcommand{\plawerr}{0.3\xspace}
\newcommand{\mmin}{\ensuremath{10^4~\msun}\xspace}
\newcommand{\kmspc}{\textrm{km~s}\ensuremath{^{-1}}\textrm{pc}\ensuremath{^{-1}}\xspace}	%  km s-1 pc-1
\newcommand{\sqcm}{cm$^{2}$\xspace}		%  cm^2
\newcommand{\percc}{\ensuremath{\textrm{cm}^{-3}}\xspace}
\newcommand{\persc}{\ensuremath{\textrm{cm}^{-2}}\xspace}
\newcommand{\persr}{\ensuremath{\textrm{sr}^{-1}}\xspace}
\newcommand{\peryr}{\ensuremath{\textrm{yr}^{-1}}\xspace}
\newcommand{\perpc}{\textrm{pc}\ensuremath{^{-1}}\xspace}
\newcommand{\perkms}{\textrm{per~km~s}\ensuremath{^{-1}}\xspace}	%  km s-1 
\newcommand{\perkmspc}{\perkms\perpc}	%  km s-1 pc-1
\newcommand{\um}{\ensuremath{\mu \textrm{m}}\xspace}    % micron
\newcommand{\mum}{\um\xspace}
\newcommand{\htwo}{\ensuremath{\textrm{H}_2}}    % micron
\newcommand{\Htwo}{\ensuremath{\textrm{H}_2}}    % micron
\newcommand{\HtwoO}{\ensuremath{\textrm{H}_2\textrm{O}}}    % micron
\newcommand{\htwoo}{\ensuremath{\textrm{H}_2\textrm{O}}}    % micron
\newcommand{\ha}{\ensuremath{\textrm{H}\alpha}}
\newcommand{\hb}{\ensuremath{\textrm{H}\beta}}
%\newcommand{\so}{ SO~(5~6)-(4~5) }
\newcommand{\regone}{Sh~2-201}
\newcommand{\regtwo}{AFGL~4029}
\newcommand{\regthree}{LW Cas Nebula}
\newcommand{\regfour}{IC 1848}
\newcommand{\regfive}{W5 NW}
\newcommand{\regsix}{SFO 11}
\newcommand{\so}{SO~\ensuremath{5_6-4_5}\xspace}
\newcommand{\SO}{SO~\ensuremath{1_2-1_1}\xspace}
\newcommand{\ammonia}{NH\ensuremath{_3}\xspace}
\newcommand{\twelveco}{\ensuremath{^{12}\textrm{CO}}\xspace}
\newcommand{\thirteenco}{\ensuremath{^{13}\textrm{CO}}\xspace}
\newcommand{\ceighteeno}{\ensuremath{\textrm{C}^{18}\textrm{O}}\xspace}
\def\ee#1{\ensuremath{\times10^{#1}}}
\newcommand{\degrees}{\ensuremath{^{\circ}}}
% can't have \degree because I'm getting a degree...
\newcommand{\lowirac}{800}
\newcommand{\highirac}{8000}
\newcommand{\lowmips}{600}
\newcommand{\highmips}{5000}
\newcommand{\perbeam}{\ensuremath{\textrm{beam}^{-1}}}
\newcommand{\ds}{\ensuremath{\textrm{d}s}}
\newcommand{\dnu}{\ensuremath{\textrm{d}\nu}}
\newcommand{\dv}{\ensuremath{\textrm{d}v}}
\def\secref#1{Section \ref{#1}}
\def\eqref#1{Equation \ref{#1}}
%\newcommand{\arcmin}{'}

\newcommand{\necluster}{Sh~2-233IR~NE}
\newcommand{\swcluster}{Sh~2-233IR~SW}
\newcommand{\region}{IRAS 05358}

\newcommand{\nwfive}{40}
\newcommand{\nouter}{15}

\newcommand{\vone}{{\rm v}1.0\xspace}
\newcommand{\vtwo}{{\rm v}2.0\xspace}
\newcommand\mjysr{\ensuremath{{\rm MJy~sr}^{-1}}}
\newcommand\jybm{\ensuremath{{\rm Jy~bm}^{-1}}}
\newcommand\nbolocat{8552\xspace}
\newcommand\nbolocatnew{548\xspace}
\newcommand\nbolocatnonew{8004\xspace} % = nbolocat-nbolocatnew
% \newcommand\arcdeg{\mbox{$^\circ$}\xspace} 
% \newcommand\arcmin{\mbox{$^\prime$}\xspace} 
% \newcommand\arcsec{\mbox{$^{\prime\prime}$}\xspace} 

\newcommand{\todo}[1]{\textcolor{red}{#1}}
\newcommand{\okinfinal}[1]{{#1}}
%% only needed if not aastex
%\newcommand{\keywords}[1]{}
%\newcommand{\email}[1]{}
%\newcommand{\affil}[1]{}


%aastex hack
%\newcommand\arcdeg{\mbox{$^\circ$}}%
%\newcommand\arcmin{\mbox{$^\prime$}\xspace}%
%\newcommand\arcsec{\mbox{$^{\prime\prime}$}\xspace}%

%\newcommand\epsscale[1]{\gdef\eps@scaling{#1}}
%
%\newcommand\plotone[1]{%
% \typeout{Plotone included the file #1}
% \centering
% \leavevmode
% \includegraphics[width={\eps@scaling\columnwidth}]{#1}%
%}%
%\newcommand\plottwo[2]{{%
% \typeout{Plottwo included the files #1 #2}
% \centering
% \leavevmode
% \columnwidth=.45\columnwidth
% \includegraphics[width={\eps@scaling\columnwidth}]{#1}%
% \hfil
% \includegraphics[width={\eps@scaling\columnwidth}]{#2}%
%}}%


%\newcommand\farcm{\mbox{$.\mkern-4mu^\prime$}}%
%\let\farcm\farcm
%\newcommand\farcs{\mbox{$.\!\!^{\prime\prime}$}}%
%\let\farcs\farcs
%\newcommand\fp{\mbox{$.\!\!^{\scriptscriptstyle\mathrm p}$}}%
%\newcommand\micron{\mbox{$\mu$m}}%
%\def\farcm{%
% \mbox{.\kern -0.7ex\raisebox{.9ex}{\scriptsize$\prime$}}%
%}%
%\def\farcs{%
% \mbox{%
%  \kern  0.13ex.%
%  \kern -0.95ex\raisebox{.9ex}{\scriptsize$\prime\prime$}%
%  \kern -0.1ex%
% }%
%}%

\def\Figure#1#2#3#4#5{
\begin{figure*}[htp]
\includegraphics[scale=#4,angle=#5]{#1}
\caption{#2}
\label{#3}
\end{figure*}
}

% originally intended to be included in a two-column paper
% this is in includegraphics: ,width=3in
% but, not for thesis
\def\OneColFigure#1#2#3#4#5{
\begin{figure}[htpb]
\epsscale{#4}
\includegraphics[scale=#4,angle=#5]{#1}
\caption{#2}
\label{#3}
\end{figure}
}

\def\SubFigure#1#2#3#4#5{
\begin{figure*}[htp]
\addtocounter{figure}{-1}
\epsscale{#4}
\includegraphics[angle=#5]{#1}
\caption{#2}
\label{#3}
\end{figure*}
}

\def\FigureTwo#1#2#3#4#5{
\begin{figure*}[htp]
\epsscale{#5}
\plottwo{#1}{#2}
\caption{#3}
\label{#4}
\end{figure*}
}

\def\FigureTwoAA#1#2#3#4#5#6{
\begin{figure*}[htp]
\subfigure[]{ \includegraphics[scale=#5,width=#6]{#1} }
\\
\subfigure[]{ \includegraphics[scale=#5,width=#6]{#2} }
\caption{#3}
\label{#4}
\end{figure*}
}


\def\TallFigureTwo#1#2#3#4#5#6{
    \FigureTwo{#1}{#2}{#3}{#4}{#5}
    }

\def\SubFigureTwo#1#2#3#4#5{
\begin{figure*}[htp]
\addtocounter{figure}{-1}
\epsscale{#5}
\plottwo{#1}{#2}
\caption{#3}
\label{#4}
\end{figure*}
}

\def\FigureFour#1#2#3#4#5#6{
\begin{figure*}[htp]
\subfigure[]{ \includegraphics[width=3in,type=png,ext=.png,read=.png]{#1} }
\subfigure[]{ \includegraphics[width=3in,type=png,ext=.png,read=.png]{#2} }
\subfigure[]{ \includegraphics[width=3in,type=png,ext=.png,read=.png]{#3} }
\subfigure[]{ \includegraphics[width=3in,type=png,ext=.png,read=.png]{#4} }
\caption{#5}
\label{#6}
\end{figure*}
}

\def\FigureThreePDF#1#2#3#4#5#6{
\begin{figure*}[htp]
\subfigure[]{ \includegraphics[width=#6,type=pdf,ext=.pdf,read=.pdf]{#1} }
\subfigure[]{ \includegraphics[width=#6,type=pdf,ext=.pdf,read=.pdf]{#2} }
\subfigure[]{ \includegraphics[width=#6,type=pdf,ext=.pdf,read=.pdf]{#3} }
\caption{#4}
\label{#5}
\end{figure*}
}

\def\FigureFourPDF#1#2#3#4#5#6#7{
\begin{figure*}[htp]
\subfigure[]{ \includegraphics[width=#7,type=pdf,ext=.pdf,read=.pdf]{#1} }
\subfigure[]{ \includegraphics[width=#7,type=pdf,ext=.pdf,read=.pdf]{#2} }
\subfigure[]{ \includegraphics[width=#7,type=pdf,ext=.pdf,read=.pdf]{#3} }
\subfigure[]{ \includegraphics[width=#7,type=pdf,ext=.pdf,read=.pdf]{#4} }
\caption{#5}
\label{#6}
\end{figure*}
}

\def\Table#1#2#3#4#5#6{
%\renewcommand{\thefootnote}{\alph{footnote}}
\begin{deluxetable}{#1}
\tablewidth{0pt}
\tabletypesize{\footnotesize}
\tablecaption{#2}
\tablehead{#3}
\startdata
\label{#4}
#5
\enddata
\bigskip
#6
\end{deluxetable}
%\renewcommand{\thefootnote}{\arabic{footnote}}
}

%\def\tablenotetext#1#2{
%\footnotetext[#1]{#2}
%}

\def\LongTable#1#2#3#4#5#6#7#8{
% required to get tablenotemark to work: http://www2.astro.psu.edu/users/stark/research/psuthesis/longtable.html
\renewcommand{\thefootnote}{\alph{footnote}}
\begin{longtable}{#1}
\caption[#2]{#2}
\label{#4} \\

 \\
\hline 
#3 \\
\hline
\endfirsthead

\hline
#3 \\
\hline
\endhead

\hline
\multicolumn{#8}{r}{{Continued on next page}} \\
\hline
\endfoot

\hline 
\endlastfoot
#7 \\

#5
\hline
#6 \\

\end{longtable}
\renewcommand{\thefootnote}{\arabic{footnote}}
}

\def\TallFigureTwo#1#2#3#4#5#6{
\begin{figure*}[htp]
\epsscale{#5}
\subfigure[]{ \includegraphics[width=#6]{#1} }
\subfigure[]{ \includegraphics[width=#6]{#2} }
\caption{#3}
\label{#4}
\end{figure*}
}

		% file containing author's macro definitions

\section{Introduction}
Turbulence is important.

Federrath, Kainulainen, Kritsuk, etc.


\section{Non-star-forming, low column-density clouds in absorption}
In \citet{Ginsburg2011a}, we noted that the \formaldehyde densitometer revealed
volume densities much higher than expected given the cloud-average densities
from \thirteenco observations.  The densities were higher even than typical
turbulence will allow.  However, this argument was made on the basis of a
statistical argument; here we attempt to demonstrate that the clumps in GMCs
are of very high density in individual clouds.

In order to detect low-column-density clouds, we must use bright background
illumination sources at 2 and 6 cm, i.e. HII regions.  There are a few dozen of
these within the inner Galactic plane, including the sources observed in
\citet{Ginsburg2011a} and the majority of the bright sources in the BGPS
\citep{Ginsburg2013}.

As an example case-study, we examine G43.17+0.01, also known as W49.  In the
large survey, we observed two lines of sight towards W49, the second at
G43.16-0.03.  Both are very bright continuum sources, and two GMCs are easily
detected in \formaldehyde absorption and \thirteenco emission.  Figure
\ref{fig:w49fullspec} shows the spectrum dominated by W49 itself, but with
clear foreground absorption components.  The continuum level subtracted from the spectra
are 73 K at 6 cm and 11 K at 2 cm for the south component, and 194 K at 6 cm
and 28 K at 2 cm for the north component.

% 2001ApJ...551..747S

\FigureTwo{figures/G43.17+0.01_H2CO_overplot_gbt9x.png}
{figures/G43.16-0.03_H2CO_overplot_gbt9x.png}
{Spectra of the \formaldehyde \oneone (black), \twotwo (red), and \thirteenco
1-0 (green) lines towards G43.17+0.01 (left) and G43.16-0.03 (right).
The \formaldehyde spectra are shown continuum-subtracted, and the \thirteenco
spectrum is offset by 1 K for clarity.  The GBT \twotwo spectra are multiplied
by a factor of 9 so the smaller lines can be seen.
}{fig:w49fullspec}{1}

We focus on the ``foreground'' lines at $\sim40$ \kms and $\sim65$ \kms, since
they are not associated with the extremely massive W49 region.  It is difficult
to assess the level of star formation within these clouds, since they lie
directly along the line of sight to W49, but additional \formaldehyde spectra
of the surrounding sources that are bright at 8-1100 \um show that they are all
at the velocity of W49 and therefore are not associated with these foreground
clouds.  Additionally, the 40 \kms cloud, known as GRSMC 43.30-0.33
\citep{Simon2001a}, was confirmed in that paper to have no associated star
formation.

The 40 \kms cloud, is observed in
its outskirts, not at the peak of the \thirteenco emission.  The cloud
structure is vast, spanning $\sim0.6\degrees$, or $\sim60$ pc at $D=2.8$ kpc
\citep{Roman-Duval2009a}.  It is detected in \oneone absorption at all 6
locations observed in \formaldehyde (Figure \ref{fig:40kmscloud}), but \twotwo
is only detected in front of the W49 HII region because of the higher
signal-to-noise at that location.  The detected \thirteenco and \formaldehyde
lines are fairly narrow, with \formaldehyde FWHM $\sim1.3$-$2.8$ \kms and
\thirteenco widths from 1.8-5.9 \kms.  The \thirteenco lines are 50\% wider
than the \formaldehyde lines.

The highest \thirteenco contours are observed as a modest IRDC, but no dust
emission peaks are observed at 500 \um or 1.1 mm.  This is an indication that
any star formation, if present, is weak - no clusters are presently forming
from this cloud.   It resembles, in that respect, the California molecular
cloud.  However, it is much smaller, with $M\approx8.3\ee{3}\pm3.2\ee{3} \msun$
compared to California's $\sim10^5$.

\Figure{figures/W49_RGB_40kms_aplpy.png}
{The G43 40 \kms cloud.  The background image shows Herschel SPIRE 70 \um (red),
Spitzer MIPS 24 \um (green), and Spitzer IRAC 8 \um (blue) in the background with
the \thirteenco integrated image from $v=36 \kms$ to $v=43 \kms$ at contour levels of
1, 2, and 3 K superposed in orange contours.  The red and black circles
show the locations of \formaldehyde pointings, and their labels indicate the LSR velocity
of the strongest line in the spectrum.  The W49 HII region is seen
behind some of the faintest \thirteenco emission that is readily associated
with this cloud.  The dark swath in the 8 and 24 \um emission going through the
peak of the \thirteenco emission in the lower half of the image is likely a low
optical depth infrared dark cloud associated with this GMC.}
{fig:40kmscloud}{0.5}{0}

The \formaldehyde densitometer measurements are shown in Figure \ref{fig:h2codensg43}.
The figures show optical depth spectra, given by the equation
$$\tau = -\log\left(\frac{S_\nu + 2.73}{\bar{C_\nu} + 2.73}\right)$$
where $S_\nu$ is the spectrum (with continuum included) and $\bar{C_\nu}$ is
the measured continuum.

\FigureTwo{figures/G43.16-0.03_40kms_h2codensfit.png}
{figures/G43.17+0.01_40kms_h2codensfit.png}
{Optical depth spectra of the \oneone and \twotwo lines towards the two W49
lines of sight, G43.16 (left) and G43.17 (right).  The fitted parameters, along with the statistical 1-$\sigma$
errors, are shown in the legend.  The optical depth ratio falls in a regime
where temperature has very little effect and there is no degeneracy between low
and high densities \citep[see Figure 2 of][]{Ginsburg2011a}.  For the right line,
it is also unaffected by lognormal turbulence, i.e. no matter what the width of
the density distribution, the measured density remains unchanged \citep[see
Figure 3 of][]{Ginsburg2011a}.}
{fig:h2codensg43}{1}

The density measurements are very precise, with $n\approx1.56\times10^4 \pm
0.14\ee{4}$ \percc and $n\approx 1.98\times10^4 \pm 0.32\ee{4}$ \percc for
G43.17+0.01 and G43.16-0.03 respectively.  At this level of precision, the 
density measurements are dominated by systematics - especially gas temperature
and collision rate uncertainties - which limit the accuracy to $\sim50\%$ using
the \citet{Green1991} rates
\citep{Zeiger2010}.  Nonetheless, the density is much higher than the
\thirteenco-measured cloud-average density $n\approx 400$ \percc
\citep[for cloud GRSMC\_G043.04-00.11;][]{Roman-Duval2010a}, with
$n_{\formaldehyde}/n_{\thirteenco} \approx 50$.  The discrepancy is worse using
the \citet{Simon2001a} cloud-averaged density $n\approx 100$ \percc.
Our density measurements are about 4$\times$ higher than CO/CI LVG density
measurements from \citet{Plume2004a}, though those measurements rely on
uncertain abundances and are fairly sensitive to temperature.

Since the W49 line of sight is clearly on the outskirts of the cloud, not
through its core, such a high density is unlikely to be an indication that
this line of sight corresponds to a centrally condensed density peak (e.g., a core).  Using
Figure 4 of \citet{Ginsburg2011a}, we can `turbulence-correct' the density
measurements, but even in the most extreme case with a turbulent density
distribution lognormal width $\sigma_s = 1.5$, the correction is only a factor
of 2.5, reducing the discrepancy to a factor of $\sim20$.

% We should then ask, if there is gas at high density, how much is at this density?
% To address this question, we'll assume that the densities in all of the \formaldehyde
% lines of sight in the cloud are the same, and compare the \thirteenco and
% \formaldehyde derived column densities.  The \oneone line robustly reflects the
% total \formaldehyde column, even though it does not constrain the density
% without a corresponding \twotwo detection.

Comparing the integrated \formaldehyde lines to the integrated \thirteenco
lines, the integrated \formaldehyde column densities are
$N_{\ortho} = 2.03\ee{12} $ and $1.56\ee{12}$ \persc for G43.16
and G43.17 respectively.
The \thirteenco integrated spectra have brightness $T_{MB} = 2.6$ K and $1.3$ K
for G43.16 and G43.17 respectively.  Using the cloud-averaged excitation
temperature for this cloud, $\tau_{13}=2.3$ and $0.6$ respectively, so
\citet{Roman-Duval2010a} equation 3 yields column densities $N_{13} = 6.2\ee{15}
$ and $1.6\ee{15}$ \percc respectively.  Assuming an abundance relative to \hh
$X_{13} = 1.8\ee{-6}$ \citep[consistent with ][]{Roman-Duval2010a}, the
resulting \hh column densities are 3.5\ee{21} and 9.0 \ee{20} \percc
respectively.  The abundances of \ortho relative to \thirteenco are 3.2\ee{-4}
and 9.8\ee{-4} respectively, or relative to \hh, 5.8\ee{-10} and 1.7\ee{-9},
which are entirely consistent with other measurements of $X_{\ortho}$.  
%These
%are relatively modest column densities, with $A_V=17$ and 4.5;
%these measurements are consistent with \citet{Plume2004a} if the different
%A_V/N(H_2) conversions are corrected.

These measurements for a specific cloud validate the statistical argument made
in \citet{Ginsburg2011a}.  However, upon closer inspection of the cloud
morphology, the real explanation may be simple: the filling factor of gas
within the GMC is small on large scales, not local scales.  The implied volume
filling factor from this analysis and the \citet{Ginsburg2011a} analysis is
$\sim10^{-2}$; the assumption of spherical symmetry is therefore extremely
poor.  

This low filling factor has major implications for the gas: if it is in
gravitational collapse, the free-fall times are shorter by an order of
magnitude than usually assumed.  The long lifetimes of GMCs therefore implies
that the cloud cannot be undergoing gravitational collapse, but instead
maintains a turbulent equilibrium.  \todo{Strengthen this argument...}

It also demonstrates that density-based star-formation thresholds do not
independently predict star formation \citep{Parmentier2011a}.  Star formation
cannot simply be driven by the free-fall time of gas, since apparently much of
the gas above $n>10^4$ \percc is not in free-fall.

% 3c111 is in california, not 3c123
% \subsection{Comparison to 3C123 and the California Nebula}
% The radio source 3C123, an active galactic nucleus, is often used as a flux
% calibrator for radio telescopes.  We used it for that purpose in our GBT
% observations, and detected the \twotwo line.  \citet{Liszt1995a} detected the
% \oneone line with the NRAO 43m telescope.  The line ratio in front of 3C123 is
% approximately $\tau_{1-1}/\tau_{2-2} \approx 10$, which indicates a density
% $n\approx10^{3.6}$ \percc.  This density is significantly lower than in the
% W49 40 \kms cloud, but still higher than expected in an inactive GMC
% \citep[which this is][]{Harvey2013a}.
%
% 3c111 may have VLBA, VLA observations


% A low filling-factor may have major impact on analyses of the distribution
% functions of column density that have recently become popular
% \citep[e.g][]{Kainulainen2009}.

\section{Implications for Turbulence}
Supersonic interstellar turbulence can be characterized by its driving mode,
Mach number $\mathcal{M}$, and magnetic field strength.  The with of the turbulent density distribution
is given by
\begin{equation}
    \sigma_s^2 = \ln\left(1+b^2 \mathcal{M}^2 \frac{\beta}{\beta+1}\right)
\end{equation}
where $\beta= 2 c_s^2/v_A^2 = 2 \mathcal{M}_A^2/\mathcal{M}^2$ and $b$ ranges
from 1/3 (solenoidal, divergence-free forcing) to 1 (compressive, curl-free)
forcing.

The observed \formaldehyde ratio depends on the \emph{mass-weighted}
probability distribution function (as opposed to the volume-weighted
distribution function, which is typically reported in simulations)
such that $p_m(s) = \rho \cdot p_s(s)$, or
\begin{equation}
    p_m(s) = \frac{\rho}{\sqrt{2 \pi \sigma_s^2}} \exp{\left(-\frac{(s-s_0)^2}{2 \sigma_s^2}\right)}
\end{equation}

We use LVG models of the \formaldehyde lines, which are computed assuming a
fixed local density, as a starting point to model the observations of
\formaldehyde in turbulence.   Starting with a fixed \emph{mean} density, we
compute the observed \formaldehyde optical depth in both the \oneone and \twotwo
line by averaging over the mass-weighted density distribution.
\begin{equation}
    \label{eqn:tauintegral}
    \tau(\bar{n}) = \int_0^\infty \tau(n) p_m(n) dn
\end{equation}
where $\tau(n)$ is computed for a given density assuming a fixed
\emph{abundance} of \ortho relative to \hh, which necessarily implies a higher
column density of \ortho for the higher densities in Equation
\ref{eqn:tauintegral}.  As long as the \formaldehyde lines are optically thin,
this approach should yield the right \emph{ratio} of the two lines, although the
absolute optical depths may be substantially smaller because of lower total
\ortho columns.

\Figure{figures/lognormalsmooth_density_ratio_massweight_logopr-1.0_abund-9.png}


\ifstandalone
\bibliographystyle{apj_w_etal}	% or "siam", or "alpha", or "abbrv"
\bibliography{thesis}		% bib database file refs.bib
\fi

\end{document}
