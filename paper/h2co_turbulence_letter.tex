%\documentclass[defaultstyle,11pt]{thesis}
%\documentclass[]{report}
%\documentclass[]{article}
%\usepackage{aastex_hack}
%\usepackage{deluxetable}
\documentclass[preprint]{aastex}
%\documentclass{emulateapj}


%%%%%%%%%%%%%%%%%%%%%%%%%%%%%%%%%%%%%%%%%%%%%%%%%%%%%%%%%%%%%%%%
%%%%%%%%%%%  see documentation for information about  %%%%%%%%%%
%%%%%%%%%%%  the options (11pt, defaultstyle, etc.)   %%%%%%%%%%
%%%%%%%  http://www.colorado.edu/its/docs/latex/thesis/  %%%%%%%
%%%%%%%%%%%%%%%%%%%%%%%%%%%%%%%%%%%%%%%%%%%%%%%%%%%%%%%%%%%%%%%%
%		\documentclass[typewriterstyle]{thesis}
% 		\documentclass[modernstyle]{thesis}
% 		\documentclass[modernstyle,11pt]{thesis}
%	 	\documentclass[modernstyle,12pt]{thesis}

%%%%%%%%%%%%%%%%%%%%%%%%%%%%%%%%%%%%%%%%%%%%%%%%%%%%%%%%%%%%%%%%
%%%%%%%%%%%    load any packages which are needed    %%%%%%%%%%%
%%%%%%%%%%%%%%%%%%%%%%%%%%%%%%%%%%%%%%%%%%%%%%%%%%%%%%%%%%%%%%%%
\usepackage{latexsym}		% to get LASY symbols
\usepackage{graphicx}		% to insert PostScript figures
%\usepackage{deluxetable}
%\usepackage{rotating}		% for sideways tables/figures
\usepackage{natbib}  % Requires natbib.sty, available from http://ads.harvard.edu/pubs/bibtex/astronat/
\usepackage{savesym}
\usepackage{amssymb}
%\savesymbol{singlespace}
\savesymbol{doublespace}
%\usepackage{wrapfig}
%\usepackage{setspace}
\usepackage{xspace}
\usepackage{color}
%\usepackage{multicol}
\usepackage{mdframed}
\usepackage{url}
\usepackage{subfigure}
%\usepackage{emulateapj}
%\usepackage{lscape}
\usepackage{grffile}
\usepackage{standalone}
\standalonetrue
\usepackage{import}
\usepackage[utf8]{inputenc}
%\usepackage{longtable}
\usepackage{booktabs}



\newcommand{\rhoPDF}{\ensuremath{\rho-\mathrm{PDF}}\xspace}
\newcommand{\meanrho}{\ensuremath{\langle\rho\rangle}\xspace}
\newcommand{\GRSMC}{GRSMC 43.30-0.33\xspace}
\newcommand{\north}{G43.17+0.01\xspace}
\newcommand{\south}{G43.16-0.03\xspace}
\newcommand{\paa}{Pa\ensuremath{\alpha}}
\newcommand{\brg}{Br\ensuremath{\gamma}}
\newcommand{\msun}{\ensuremath{M_{\odot}}\xspace}			%  Msun
\newcommand{\mdot}{\ensuremath{\dot{M}}\xspace}
\newcommand{\lsun}{\ensuremath{L_{\odot}}}			%  Lsun
\newcommand{\lbol}{\ensuremath{L_{\mathrm{bol}}}}	%  Lbol
\newcommand{\ks}{K\ensuremath{_{\mathrm{s}}}}		%  Ks
\newcommand{\hh}{\ensuremath{\textrm{H}_{2}}\xspace}			%  H2
\newcommand{\formaldehyde}{\ensuremath{\textrm{H}_2\textrm{CO}}\xspace}
\newcommand{\formaldehydeIso}{\ensuremath{\textrm{H}_2~^{13}\textrm{CO}}\xspace}
\newcommand{\methanol}{\ensuremath{\textrm{CH}_3\textrm{OH}}\xspace}
\newcommand{\ortho}{\ensuremath{\textrm{o-H}_2\textrm{CO}}\xspace}
\newcommand{\oneone}{\ensuremath{1_{10}-1_{11}}\xspace}
\newcommand{\twotwo}{\ensuremath{2_{11}-2_{12}}\xspace}
\newcommand{\threethree}{\ensuremath{3_{12}-3_{13}}\xspace}
\newcommand{\threeohthree}{\ensuremath{3_{03}-2_{02}}\xspace}
\newcommand{\threetwotwo}{\ensuremath{3_{22}-2_{21}}\xspace}
\newcommand{\threetwoone}{\ensuremath{3_{21}-2_{20}}\xspace}
\newcommand{\JKaKc}{\ensuremath{J_{K_a K_c}}}
\newcommand{\water}{H$_{2}$O}		%  H2O
\newcommand{\feii}{\ion{Fe}{2}}		%  FeII
\newcommand{\uchii}{UC\ion{H}{2}\xspace}
\newcommand{\UCHII}{UC\ion{H}{2}\xspace}
\newcommand{\hii}{H~{\sc ii}\xspace}
\newcommand{\Hii}{H~{\sc ii}\xspace}
\newcommand{\HII}{H~{\sc ii}\xspace}
\newcommand{\kms}{\textrm{km~s}\ensuremath{^{-1}}\xspace}	%  km s-1
\newcommand{\nsample}{456\xspace}
\newcommand{\CFR}{5\xspace} % nMPC / 0.25 / 2 (6 for W51 once, 8 for W51 twice) REFEDIT: With f_observed=0.3, becomes 3/2./0.3 = 5
\newcommand{\permyr}{\ensuremath{\mathrm{Myr}^{-1}}\xspace}
\newcommand{\tsuplim}{0.5\xspace} % upper limit on starless timescale
\newcommand{\ncandidates}{18\xspace}
\newcommand{\mindist}{8.7\xspace}
\newcommand{\rcluster}{2.5\xspace}
\newcommand{\ncomplete}{13\xspace}
\newcommand{\middistcut}{13.0\xspace}
\newcommand{\nMPC}{3\xspace} % only count W51 once.  W51, W49, G010
\newcommand{\obsfrac}{30}
\newcommand{\nMPCtot}{10\xspace} % = nmpc / obsfrac
\newcommand{\nMPCtoterr}{6\xspace} % = sqrt(nmpc) / obsfrac
\newcommand{\plaw}{2.1\xspace}
\newcommand{\plawerr}{0.3\xspace}
\newcommand{\mmin}{\ensuremath{10^4~\msun}\xspace}
\newcommand{\kmspc}{\textrm{km~s}\ensuremath{^{-1}}\textrm{pc}\ensuremath{^{-1}}\xspace}	%  km s-1 pc-1
\newcommand{\sqcm}{cm$^{2}$\xspace}		%  cm^2
\newcommand{\percc}{\ensuremath{\textrm{cm}^{-3}}\xspace}
\newcommand{\persc}{\ensuremath{\textrm{cm}^{-2}}\xspace}
\newcommand{\persr}{\ensuremath{\textrm{sr}^{-1}}\xspace}
\newcommand{\peryr}{\ensuremath{\textrm{yr}^{-1}}\xspace}
\newcommand{\perpc}{\textrm{pc}\ensuremath{^{-1}}\xspace}
\newcommand{\perkms}{\textrm{per~km~s}\ensuremath{^{-1}}\xspace}	%  km s-1 
\newcommand{\perkmspc}{\perkms\perpc}	%  km s-1 pc-1
\newcommand{\um}{\ensuremath{\mu \textrm{m}}\xspace}    % micron
\newcommand{\mum}{\um\xspace}
\newcommand{\htwo}{\ensuremath{\textrm{H}_2}}    % micron
\newcommand{\Htwo}{\ensuremath{\textrm{H}_2}}    % micron
\newcommand{\HtwoO}{\ensuremath{\textrm{H}_2\textrm{O}}}    % micron
\newcommand{\htwoo}{\ensuremath{\textrm{H}_2\textrm{O}}}    % micron
\newcommand{\ha}{\ensuremath{\textrm{H}\alpha}}
\newcommand{\hb}{\ensuremath{\textrm{H}\beta}}
%\newcommand{\so}{ SO~(5~6)-(4~5) }
\newcommand{\regone}{Sh~2-201}
\newcommand{\regtwo}{AFGL~4029}
\newcommand{\regthree}{LW Cas Nebula}
\newcommand{\regfour}{IC 1848}
\newcommand{\regfive}{W5 NW}
\newcommand{\regsix}{SFO 11}
\newcommand{\so}{SO~\ensuremath{5_6-4_5}\xspace}
\newcommand{\SO}{SO~\ensuremath{1_2-1_1}\xspace}
\newcommand{\ammonia}{NH\ensuremath{_3}\xspace}
\newcommand{\twelveco}{\ensuremath{^{12}\textrm{CO}}\xspace}
\newcommand{\thirteenco}{\ensuremath{^{13}\textrm{CO}}\xspace}
\newcommand{\ceighteeno}{\ensuremath{\textrm{C}^{18}\textrm{O}}\xspace}
\def\ee#1{\ensuremath{\times10^{#1}}}
\newcommand{\degrees}{\ensuremath{^{\circ}}}
% can't have \degree because I'm getting a degree...
\newcommand{\lowirac}{800}
\newcommand{\highirac}{8000}
\newcommand{\lowmips}{600}
\newcommand{\highmips}{5000}
\newcommand{\perbeam}{\ensuremath{\textrm{beam}^{-1}}}
\newcommand{\ds}{\ensuremath{\textrm{d}s}}
\newcommand{\dnu}{\ensuremath{\textrm{d}\nu}}
\newcommand{\dv}{\ensuremath{\textrm{d}v}}
\def\secref#1{Section \ref{#1}}
\def\eqref#1{Equation \ref{#1}}
%\newcommand{\arcmin}{'}

\newcommand{\necluster}{Sh~2-233IR~NE}
\newcommand{\swcluster}{Sh~2-233IR~SW}
\newcommand{\region}{IRAS 05358}

\newcommand{\nwfive}{40}
\newcommand{\nouter}{15}

\newcommand{\vone}{{\rm v}1.0\xspace}
\newcommand{\vtwo}{{\rm v}2.0\xspace}
\newcommand\mjysr{\ensuremath{{\rm MJy~sr}^{-1}}}
\newcommand\jybm{\ensuremath{{\rm Jy~bm}^{-1}}}
\newcommand\nbolocat{8552\xspace}
\newcommand\nbolocatnew{548\xspace}
\newcommand\nbolocatnonew{8004\xspace} % = nbolocat-nbolocatnew
% \newcommand\arcdeg{\mbox{$^\circ$}\xspace} 
% \newcommand\arcmin{\mbox{$^\prime$}\xspace} 
% \newcommand\arcsec{\mbox{$^{\prime\prime}$}\xspace} 

\newcommand{\todo}[1]{\textcolor{red}{#1}}
\newcommand{\okinfinal}[1]{{#1}}
%% only needed if not aastex
%\newcommand{\keywords}[1]{}
%\newcommand{\email}[1]{}
%\newcommand{\affil}[1]{}


%aastex hack
%\newcommand\arcdeg{\mbox{$^\circ$}}%
%\newcommand\arcmin{\mbox{$^\prime$}\xspace}%
%\newcommand\arcsec{\mbox{$^{\prime\prime}$}\xspace}%

%\newcommand\epsscale[1]{\gdef\eps@scaling{#1}}
%
%\newcommand\plotone[1]{%
% \typeout{Plotone included the file #1}
% \centering
% \leavevmode
% \includegraphics[width={\eps@scaling\columnwidth}]{#1}%
%}%
%\newcommand\plottwo[2]{{%
% \typeout{Plottwo included the files #1 #2}
% \centering
% \leavevmode
% \columnwidth=.45\columnwidth
% \includegraphics[width={\eps@scaling\columnwidth}]{#1}%
% \hfil
% \includegraphics[width={\eps@scaling\columnwidth}]{#2}%
%}}%


%\newcommand\farcm{\mbox{$.\mkern-4mu^\prime$}}%
%\let\farcm\farcm
%\newcommand\farcs{\mbox{$.\!\!^{\prime\prime}$}}%
%\let\farcs\farcs
%\newcommand\fp{\mbox{$.\!\!^{\scriptscriptstyle\mathrm p}$}}%
%\newcommand\micron{\mbox{$\mu$m}}%
%\def\farcm{%
% \mbox{.\kern -0.7ex\raisebox{.9ex}{\scriptsize$\prime$}}%
%}%
%\def\farcs{%
% \mbox{%
%  \kern  0.13ex.%
%  \kern -0.95ex\raisebox{.9ex}{\scriptsize$\prime\prime$}%
%  \kern -0.1ex%
% }%
%}%

\def\Figure#1#2#3#4#5{
\begin{figure*}[htp]
\includegraphics[scale=#4,angle=#5]{#1}
\caption{#2}
\label{#3}
\end{figure*}
}

% originally intended to be included in a two-column paper
% this is in includegraphics: ,width=3in
% but, not for thesis
\def\OneColFigure#1#2#3#4#5{
\begin{figure}[htpb]
\epsscale{#4}
\includegraphics[scale=#4,angle=#5]{#1}
\caption{#2}
\label{#3}
\end{figure}
}

\def\SubFigure#1#2#3#4#5{
\begin{figure*}[htp]
\addtocounter{figure}{-1}
\epsscale{#4}
\includegraphics[angle=#5]{#1}
\caption{#2}
\label{#3}
\end{figure*}
}

\def\FigureTwo#1#2#3#4#5{
\begin{figure*}[htp]
\epsscale{#5}
\plottwo{#1}{#2}
\caption{#3}
\label{#4}
\end{figure*}
}

\def\FigureTwoAA#1#2#3#4#5#6{
\begin{figure*}[htp]
\subfigure[]{ \includegraphics[scale=#5,width=#6]{#1} }
\\
\subfigure[]{ \includegraphics[scale=#5,width=#6]{#2} }
\caption{#3}
\label{#4}
\end{figure*}
}


\def\TallFigureTwo#1#2#3#4#5#6{
    \FigureTwo{#1}{#2}{#3}{#4}{#5}
    }

\def\SubFigureTwo#1#2#3#4#5{
\begin{figure*}[htp]
\addtocounter{figure}{-1}
\epsscale{#5}
\plottwo{#1}{#2}
\caption{#3}
\label{#4}
\end{figure*}
}

\def\FigureFour#1#2#3#4#5#6{
\begin{figure*}[htp]
\subfigure[]{ \includegraphics[width=3in,type=png,ext=.png,read=.png]{#1} }
\subfigure[]{ \includegraphics[width=3in,type=png,ext=.png,read=.png]{#2} }
\subfigure[]{ \includegraphics[width=3in,type=png,ext=.png,read=.png]{#3} }
\subfigure[]{ \includegraphics[width=3in,type=png,ext=.png,read=.png]{#4} }
\caption{#5}
\label{#6}
\end{figure*}
}

\def\FigureThreePDF#1#2#3#4#5#6{
\begin{figure*}[htp]
\subfigure[]{ \includegraphics[width=#6,type=pdf,ext=.pdf,read=.pdf]{#1} }
\subfigure[]{ \includegraphics[width=#6,type=pdf,ext=.pdf,read=.pdf]{#2} }
\subfigure[]{ \includegraphics[width=#6,type=pdf,ext=.pdf,read=.pdf]{#3} }
\caption{#4}
\label{#5}
\end{figure*}
}

\def\FigureFourPDF#1#2#3#4#5#6#7{
\begin{figure*}[htp]
\subfigure[]{ \includegraphics[width=#7,type=pdf,ext=.pdf,read=.pdf]{#1} }
\subfigure[]{ \includegraphics[width=#7,type=pdf,ext=.pdf,read=.pdf]{#2} }
\subfigure[]{ \includegraphics[width=#7,type=pdf,ext=.pdf,read=.pdf]{#3} }
\subfigure[]{ \includegraphics[width=#7,type=pdf,ext=.pdf,read=.pdf]{#4} }
\caption{#5}
\label{#6}
\end{figure*}
}

\def\Table#1#2#3#4#5#6{
%\renewcommand{\thefootnote}{\alph{footnote}}
\begin{deluxetable}{#1}
\tablewidth{0pt}
\tabletypesize{\footnotesize}
\tablecaption{#2}
\tablehead{#3}
\startdata
\label{#4}
#5
\enddata
\bigskip
#6
\end{deluxetable}
%\renewcommand{\thefootnote}{\arabic{footnote}}
}

%\def\tablenotetext#1#2{
%\footnotetext[#1]{#2}
%}

\def\LongTable#1#2#3#4#5#6#7#8{
% required to get tablenotemark to work: http://www2.astro.psu.edu/users/stark/research/psuthesis/longtable.html
\renewcommand{\thefootnote}{\alph{footnote}}
\begin{longtable}{#1}
\caption[#2]{#2}
\label{#4} \\

 \\
\hline 
#3 \\
\hline
\endfirsthead

\hline
#3 \\
\hline
\endhead

\hline
\multicolumn{#8}{r}{{Continued on next page}} \\
\hline
\endfoot

\hline 
\endlastfoot
#7 \\

#5
\hline
#6 \\

\end{longtable}
\renewcommand{\thefootnote}{\arabic{footnote}}
}

\def\TallFigureTwo#1#2#3#4#5#6{
\begin{figure*}[htp]
\epsscale{#5}
\subfigure[]{ \includegraphics[width=#6]{#1} }
\subfigure[]{ \includegraphics[width=#6]{#2} }
\caption{#3}
\label{#4}
\end{figure*}
}

		% file containing author's macro definitions

\begin{document}
\ifstandalone
\title{Molecular Density Measurements with \formaldehyde: Turbulence is
    Compressively Driven}
\begin{abstract}
    Molecular clouds are supersonically turbulent.  This turbulence may govern
    the form of the initial mass function and the star formation rate of the
    gas.  It is therefore essential to understand the properties of turbulence,
    in particular the probability distribution of density in turbulent clouds.

    We present \formaldehyde volume density measurements of a non-star-forming
    cloud along the line of sight towards W49A.  This method is complementary
    to measurements of turbulence via the column density distribution and
    should be applicable to any molecular cloud with detected CO.  We show that
    turbulence in this cloud must be compressively driven, with a
    compressive-to-total ratio $b = \mathcal{M}_C/\mathcal{M}>0.6$, and that a
    lognormal distribution is unlikely to describe the cloud's density.  
\end{abstract}

\fi

\todo{Red text indicates ``todo'' items}

\todo{Use $\rho$ and $n(\hh)$ consistently}

\section{Introduction}
Nearly all gas in the interstellar medium is supersonically turbulent.  The
properties of this turbulence, most importantly the shape of the density
probability distribution function (\rhoPDF), are essential for determining how star
formation progresses.
There are now predictive theories of star formation that include formulations
of the Initial Mass Function \citep[IMF;][]{Padoan2002a, Padoan2007a, Chabrier2010a, 
Elmegreen2011a, Hopkins2012b, Hennebelle2013a} and the star
formation rate
\citep[SFR;][]{Krumholz2005c,  Hennebelle2011a,
Padoan2011b, Krumholz2012b, Federrath2012a, Padoan2012a}.
The distribution of stellar masses and the overall star formation rate depend
critically on the \rhoPDF established by turbulence.  It is therefore essential to
measure the \rhoPDF in the molecular clouds that produce stars.
% Moreover, measuring the actual density of a molecular cloud is crucial, because
% the mass and gravitational energy are directly proportional to the mean density
% of the cloud, which in turn determine the virial parameter and thus the ability
% of a cloud to form low- and potentially high-mass stars.

Recent works have used simulations to characterize the density distribution
from different driving modes of turbulence
\citep{Federrath2008a,Federrath2009a,Federrath2010a,Federrath2011a,Price2011b,Federrath2013a}.
These studies determined that there is a relation between the mode of turbulent driving and the width
of the lognormal density distribution
\citep{Padoan2011b,Molina2012a}.
They characterize the turbulent driver in terms of its solenoidal (curly) versus compressive (divergent)
driving properties. \todo{Clarify this or thin it out; true description comes later?}
% Save for later
% , with the lognormal width $\sigma_{s}^2 = \ln\left(1+b^2
% \mathcal{M}^2 \frac{\beta}{\beta+1}\right)$, where $\beta=2 (\mathcal{M}_A /
% \mathcal{M})^2 = 2 (c_s/v_A)^2$ with sound speed $c_s$ and Alfven speed $v_A$,
% and the logarithmic density contrast $s\equiv\ln(\rho/\rho_0)$
% This equation can also be expressed in terms of the compressive Mach number
% $\mathcal{M}_c = b \mathcal{M}$, with $b\approx 1/3$ corresponding to
% solenoidal forcing and $b = 1$ corresponding to purely compressive forcing
% \citep{Federrath2010a,Konstandin2012a}.
%However, \citet{Hopkins2013a} notes
%that the lognormal approximation of the turbulent density distribution

All of the above turbulence-based theories of star formation explicitly assume a
lognormal form for the density probability distribution $P_V(s)$ of the gas.
However, recent simulations \citep{Kritsuk2007a, Schmidt2009a, Federrath2010a, Federrath2013a, Federrath2013b} and theoretical work
\citep{Hopkins2013a} have shown that the assumption of a
lognormal distribution is often very poor; theoretical intermittent distributions and 
simulated \rhoPDF's deviate from lognormal by orders of magnitude
at the extremes of the density distributions.  Since these theories all involve
an integral over the density probability distribution function (PDF), deviation from
the lognormal distribution can drastically affect the overall predicted star formation
rate \citep[e.g.]{Cho2011a,Collins2012a} and initial mass function.
Note that the modifications to the \rhoPDF driven by gravitational collapse
do not change the SFR or the IMF
\citep{Klessen2000a,Kritsuk2011a,Federrath2013a} since gravitational overdensities have
already separated from the turbulent flow that created them. It is therefore
crucial that studies of turbulence focus on clouds which are not yet dominated
by gravitational collapse.

While simulations are powerful probes of wide ranges of parameter space, no
simulation is capable of including all of the physical processes and spatial
scales relevant to turbulence.  Observations are required to provide additional
constraints on properties of interstellar turbulence and guide simulators
toward the most useful conditions and processes to include.
\citet{Brunt2010c}, \citet{Kainulainen2012a} and \citet{Kainulainen2013a}
provide some of the first observational constraints on the mode of turbulent
driving using extinction-derived column density distributions.
They find $b\approx0.4-0.5$, i.e. that there is a `natural' mix of solenoidal
and compressive modes.  A `natural' mixture (a 2:1 mixture) of solenoidal and
compressive modes injected by the turbulent driver, i.e., a forcing ratio
$F_{comp}/F_{sol} = 1/2$, yields $b\sim0.4$. Thus, $b>0.4$ implies an enhanced
compressive forcing component relative to the naturally mixed case \citep[see
Figure 8 in][]{Federrath2010a}.
% However, these observations still attempted to characterize a lognormal
% distribution.


Formaldehyde, \formaldehyde, is a unique probe of density in molecular clouds.
Like CO, it is ubiquitous, with a nearly constant abundance wherever CO is
found \citep{Mangum1993a,Tang2013a}.  The lowest rotational transitions of
\ortho at 2 and 6 cm can be observed in absorption against the cosmic microwave
background or any bright continuum source \citep{Ginsburg2011a,Darling2012b}.
The ratio of these lines is strongly sensitive to the local density of \hh, but
it is relatively insensitive to the local gas temperature
\citep{Troscompt2009a,Wiesenfeld2013a}.  The \formaldehyde line ratio has a
direct dependence on the density that is nearly independent of the column
density, which is unlike typical methods of molecular-line based density
inference where the density is assumed to be greater than the critical density
of the detected transition \todo{rephrase}.

However, the particular property of the \formaldehyde densitometer we exploit
here is its ability to trace the \emph{mass-weighted} density of the gas.
Typical density measurements from \thirteenco or dust measure the total mass
and assume a line-of-sight geometry, measuring a \emph{volume-weighted}
density, i.e. $<\rho>_V = M_{tot}/V_{tot}$.  In contrast, the \formaldehyde
densitometer is sensitive to the density at which most mass resides.
% - i.e. $<\rho>_M = \int M \rho(M) d M / M_{tot}$\todo{check this}.  
The volume- and mass- weighted densities have different dependencies on the
underlying density distributions, so in clouds dominated by turbulence, if we
have measurements of both, we can constrain the shape of the \rhoPDF and
perhaps the driving mode.
% TODO: Christoph had questions about this.  "We should be more precise here.
% Can you elaborate a bit on what you do when you compare rho_V with rho_M to
% get the mode mixture of the driver? Also, note that we cannot really get the
% driver (e.g., supernova, jets, callapse, HII region, ...), but only the mode
% mixture that this driver corresponds to (or excites), i.e., we can possibly
% get the b parameter, but not the phyiscal source of the turbulence (the
% actual driver)."

In \citet{Ginsburg2011a}, we noted that the \formaldehyde densitometer revealed
densities much higher than expected given the cloud-average densities from
\thirteenco observations.  The densities were higher even than typical
turbulence will allow.  However, this argument was made on the basis of a
statistical comparison of ``cloud-average'' versus \formaldehyde-based density
measurements and left open the possibility that we had selected especially
dense clouds.  In this Letter, we use the example of a single cloud to
demonstrate that the high \formaldehyde densities must be caused by the shape
of the density distribution and infer the shape of this distribution.


\section{Observations}
We report \formaldehyde observations performed at the Arecibo Radio
Observatory\footnote{The Arecibo Observatory is part of the National Astronomy
and Ionosphere Center, which is operated by Cornell University under a
cooperative agreement with the National Science Foundation.  } and the Green
Bank Telescope\footnote{ The National Radio Astronomy Observatory operates the
GBT and VLA and is a facility of the National Science Foundation operated under
cooperative agreement by Associated Universities, Inc.  } that have been 
described in more detail in \citet{Ginsburg2011a}, with additional data to be
published in a future work \todo{determine appropriate citation?}.  Arecibo and
the GBT have FWHM$\approx50$\arcsec beams at the observed frequencies of 4.829
and 14.488 GHz respectively.  Observations were carried out in a
position-switched mode with 3 and 5.5\arcmin offsets for the Arecibo and GBT
observations respectively.

The Boston University / Five-College Radio Astronomy Observatory Galactic Ring
Survey \thirteenco data was also used.  The BU FCRAO GRS \citep{Jackson2006a}
is a survey of the Galactic plane in the \thirteenco\ 1-0 line with $\sim
46\arcsec$ resolution.  We used reduced data cubes of the $\ell=43$ region.

\subsection{\GRSMC A non-star-forming molecular cloud}

% In order to detect low-column-density clouds, we must use bright background
% illumination sources at 2 and 6 cm, i.e. HII regions.  There are a few dozen of
% these within the inner Galactic plane, including the sources observed in
% \citet{Ginsburg2011a} and the majority of the bright sources in the BGPS
% \citep{Ginsburg2013b}.


We examine the line of sight toward \north, also known as W49A.  In a large
survey, we observed two lines of sight toward W49, the second at \south.  Both
are very bright radio continuum sources, and two foreground GMCs are easily
detected in both \formaldehyde absorption and \thirteenco emission.  Figure
\ref{fig:w49fullspec} shows the spectrum dominated by W49 itself, but with
clear \formaldehyde foreground absorption components.  The continuum levels
subtracted from the spectra are 73 K at 6 cm and 11 K at 2 cm for the south
component (\south), and 194 K at 6 cm and 28 K at 2 cm for the north component
(\north).

% 2001ApJ...551..747S

%codes/fit_small_lines_G43.py
%fit_small_lines.py
\FigureTwo{figures/G43.17+0.01_H2CO_overplot_gbt9x.pdf}
          {figures/G43.16-0.03_H2CO_overplot_gbt9x.pdf}
{Spectra of the \formaldehyde \oneone (black), \twotwo (red), and \thirteenco
1-0 (green) lines toward G43.17+0.01 (left) and G43.16-0.03 (right).
The \formaldehyde spectra are shown continuum-subtracted, and the \thirteenco
spectrum is offset by +1 K for clarity.  The GBT \twotwo spectra are multiplied
by a factor of 9 so the smaller lines can be seen.  The blue arrow marks the 40 \kms
cloud \GRSMC that we focus on in this paper. \todo{CUT for letter-form}
}{fig:w49fullspec}{1}

We focus on the ``foreground'' line at $\sim40$ \kms, since it is not
associated with the extremely massive W49 region (which is dominated by gravity
and stellar feedback rather than turbulence).  The cloud is shown in Figure
\ref{fig:40kmscloud}.  
% The cloud, known as GRSMC 43.30-0.33 \citep{Simon2001a},
% was confirmed to have no associated star formation in that work.  
Additional \formaldehyde spectra of surrounding sources that are bright at
8-1100 \um and within the \thirteenco contours of the cloud show that they are
all at the velocity of W49 and therefore are not associated with these
foreground clouds.  

The \formaldehyde lines are observed in the outskirts of the cloud, not at
the peak of the \thirteenco emission.  The cloud spans $\sim0.6\degrees$, or
$\sim30$ pc at $D=2.8$ kpc \citep{Roman-Duval2009a}.  It is detected in \oneone
absorption at all 6 locations observed in \formaldehyde (Figure
\ref{fig:40kmscloud}), but \twotwo is only detected in front of the W49 HII
region because of the higher signal-to-noise at that location.  The detected
\thirteenco and \formaldehyde lines are fairly narrow, with \formaldehyde FWHM
ranging from $\Delta v \sim1.3-2.8$ \kms and \thirteenco widths from $\Delta v \sim1.8-5.9$
\kms.  The \thirteenco lines are 50-100\% wider than the \formaldehyde lines.

The highest \thirteenco contours are observed as a modest infrared dark cloud
in Spitzer 8 \um images, but no dust emission peaks are observed at 500 \um
\citep[Herschel;][]{Traficante2011a} or 1.1 mm
\citep[Bolocam;][]{Aguirre2011a,Ginsburg2013b} associated with the dark gas.
This is an indication that the cloud is not dominated by gravity -- no
massive dense clumps are present within this cloud.


%
% Full GRSMC GLON deg GLAT deg VLSR km/s DelV km/s Rad pc Mass Msun e_ Msun nH2 cm-3 Tex K tau Sigma Msun/pc2 alpha Note RD09 _RA.icrs deg _DE.icrs deg
% 1	G043.14-00.36	043.14	-00.36	41.17	3.13	3.9	6.8e+03	2.2e+03	431.4	5.66	1.92	144.8	0.91	i	RD09	287.88	+08.91
% 2	G043.04-00.11	043.04	-00.11	41.59	3.48	4.2	8.3e+03	3.2e+03	394.6	5.68	1.77	145.7	1.02	i	RD09	287.61	+08.94
% 3	G043.14-00.76	043.14	-00.76	59.02	2.92	9.8	3.0e+04	1.0e+04	117.6	5.78	1.28	100.4	0.45	i	RD09	288.24	+08.72
% 4	G043.49-00.71	043.49	-00.71	41.59	1.84	1.9	1.3e+03	4.6e+02	645.7	5.23	1.85	108.8	0.84	i	RD09	288.35	+09.06
%                                   ^^^^^ WRONG!?!?!
% +09 01	19 11.5	G043.24-00.31	043.24	-00.31	 40.7	 3.11	 1.50	43.36	-0.24	 0.36	 0.26	-46	 0.13	 1.01	 8.5	 0.24	 4.2e+21	  	287.88	+09.02	  	G043.24-00.31	c1	043.79	-00.13	 43.3	 0.02	 0.01	 1.11	 7.75	 6.81e-04	 8.09	 2.78e-03	 3.81e+21	  	287.97	+09.59	  	  	  	  	  	  	  	  	  	  	  	  	  	  	  	  	  	  	  	  	  	  	  	  	  	  	  	  	RD09	  	  
% +09 01	19 11.5	G043.24-00.31	043.24	-00.31	 40.7	 3.11	 1.50	43.36	-0.24	 0.36	 0.26	-46	 0.13	 1.01	 8.5	 0.24	 4.2e+21	  	287.88	+09.02	  	G043.24-00.31	c2	043.23	-00.31	 40.5	 0.08	 0.12	 1.12	 5.00	 9.68e-03	 5.71	 1.62e-02	 2.46e+21	XY	287.87	+09.01	  	  	  	  	  	  	  	  	  	  	  	  	  	  	  	  	  	  	  	  	  	  	  	  	  	  	  	  	RD09	  	  
%                                   
%
% 6.8+8.3 = 15.1 x10^3 msun
% circle is closer to 0.3 degrees, radius=14.66 pc (0.3 * 3600 * 2800 / 206265.)
% In [105]: 1.5e4 * 2e33 / (2.8*1.67e-24) / (4/3.*pi*(15*3.08e18)**3)
% Out[105]: 15.532172896708314
% 
% In [106]: 1.5e4 * 2e33 / (2.8*1.67e-24) / ((2*15*3.08e18)**3)
% Out[106]: 8.132626711097554
% 
% In [107]: 1.5e4 * 2e33 / (2.8*1.67e-24) / (4/3.*pi*(15*3.08e18)**3*(1*1*0.1))
% Out[107]: 155.32172896708315
%
% If we use just the northern cloud, G43.04-0.11, with the Roman-Duval derived r_eff = 4.2 pc and mass 830 Msun
% tex=5.68
% tau=1.77
% colco=2.6e14 * tex * tau * (1-np.exp(-5.3/tex))
% colh2=colco * 45 /8e-5 
% npix = 11472
% areacm = npix * (22.5*2800/206265.*3.08e18)**2 
% mh2 = areacm * colh2 * 2.72 * 1.67e-24 / 2e33
% reff = (npix * (22.5*2800/206265.)**2)**0.5/np.pi
% # ALTERNATIVE:  ellipse(43.086547,-0.028025374,684.15208",451.3206",346.63003)
% r1 = 451. * 2800/206265.
% r2 = 684. * 2800/206265.
% A = r1*r2*pi
% nh2 = mh2 * 2e33 / (2.72*1.67e-24) / (4/3.*pi*(reff*3.08e18)**3)
% # = 65 cm^-3
% nh2b = mh2 * 2e33 / (2.72*1.67e-24) / (4/3.*pi*(r1*r2*r1)*(3.08e18)**3)
% # = 212 cm^-3
%
% The actual integrated 13CO for the ellipse reported above:
% In [226]: np.nansum(cld.flat[inds2a])
% Out[226]: 381209.47
% In [234]: inds2a.sum()
% Out[234]: 11472
% # mean integral:
% co =  np.nansum(cld.flat[inds2a]) / 11472 
% co_tau = -np.log(1+ (co*0.189)/((np.exp(5.3/tex) - 1)**(-1)-0.16) * -1)
% 
% 
The cloud has mass in the range $M_{CO} = 1-3\ee{4}$ \msun in a radius $r=15$ pc as measured
from the integrated \thirteenco map using an optical depth estimate and
abundance from \citet{Roman-Duval2010a}, so its mean density is $n(\hh) \approx
10-30$ \percc assuming spherical symmetry.  If we
instead assume a cubic volume, as is done in simulations, the mean density is lower. 
\citet{Simon2001a} report a mass $M_{CO} = 6\ee{4} \msun$ and $r=13$ pc,
yielding a density $n(\hh)=100$ \percc, which is consistent with our estimates.
\citet{Roman-Duval2010a} break the cloud apart into 3 separate objects for their
analysis, GRSMC 43.04-0.11, GRSMC 43.24-00.31, and GRSMC 43.14-0.36.  All three
have the same velocity to within 1 \kms, but they show slight discontinuities
in position-velocity space.  These discontinuities are morphologically consistent
with gaps seen in turbulent simulations, validating our assessment of the cloud as a
single object, but as a maximally conservative estimate we use the density of the
northmost ``clump'' GRSMC 43.04-0.11, which overlaps our target line of sight,
as an upper limit.  It has density $n\approx120\percc$, but we use $n<200$ \percc
as a slightly more conservative limit to allow for modest uncertainties in
optical depth, radius, and abundance.

%but somewhat higher than measured by \citet{Roman-Duval2010a} because of the
%improved optical depth corrections in the latter work.%
%\footnote{\citet{Roman-Duval2010a} do not report on cloud \GRSMC, but instead
%identify two ``clumps'' within the cloud, G043.14-00.36 and G043.04-00.11,
%which are each assumed to have lower mass, much smaller radii, and therefore
%higher densities.  Figure \ref{fig:40kmscloud} shows that they are indeed members
%of a single contiguous cloud.}


% It resembles, in that respect, the California molecular
% cloud.  However, it is much smaller, with $M\approx8.3\ee{3}\pm3.2\ee{3} \msun$
% compared to California's $\sim10^5$.

% G43.17+0.01/cloud_figures.py
\Figure{figures/W49_RGB_40kms_aplpy.png}
{\textit{The GRSMC 43.30-0.33 cloud.}  The background image shows Herschel SPIRE 70 \um (red),
Spitzer MIPS 24 \um (green), and Spitzer IRAC 8 \um (blue) in the background with
the \thirteenco integrated image from $v_{LSR}=36$ \kms to $v_{LSR}=43$ \kms at contour levels of
1, 2, and 3 K \kms superposed in cyan contours.  The red and black circles
show the locations of \formaldehyde pointings, and their labels indicate the LSR velocity
of the strongest line in the spectrum.  The W49 HII region is seen
behind some of the faintest \thirteenco emission that is readily associated
with this cloud.  The dark swath in the 8 and 24 \um emission going through the
peak of the \thirteenco emission in the lower half of the image is a low
optical depth infrared dark cloud associated with this GMC.}
{fig:40kmscloud}{0.5}{0}

\section{Modeling \formaldehyde}
In order to infer densities using the \formaldehyde densitometer, we use the
low-temperature collision rates given by \citet{Troscompt2009a} with RADEX
\citep{van-der-Tak2007a} to build a grid of predicted line properties covering
100 densities $n(\hh) = 10-10^8$ \percc, 10 temperatures $T=5-50$ K, 100 column densities
$N(\ortho) = 10^{11}-10^{16}$ \persc, and 10 ortho-to-para ratios 
$OPR = 0.001-3.0$.

The \formaldehyde densitometer measurements are shown in Figure \ref{fig:h2codensg43}.
The figures show optical depth spectra, given by the equation
\begin{equation}
    \tau = -\ln\left(\frac{S_\nu + 2.73\mathrm{~K}}{\bar{C_\nu} + 2.73\mathrm{~K}}\right)
\end{equation}
where $S_\nu$ is the spectrum (with both the line and continuum included) and $\bar{C_\nu}$ is
the measured continuum, both in Kelvin.  The cosmic microwave background
temperature is added to the continuum since \formaldehyde can be seen in
absorption against it, though toward W49 it is negligible.

% G43.17
% [Param #0     DENSITY0 =      4.36419 +/-       0.0755311   Range:     [1,8],
%  Param #1      COLUMN0 =      12.4276 +/-       0.0417072   Range:   [11,16],
%  Param #2   ORTHOPARA0 =     -1.25514 +/-         1.30736   Range:[-3,0.477121],
%  Param #3 TEMPERATURE0 =      27.5313 +/-         18.9722   Range:    [5,55],
%  Param #4      CENTER0 =      39.5386 +/-      0.00108955 ,
%  Param #5       WIDTH0 =     0.379159 +/-     0.000709161   Range:   [0,inf)]
% stats_dict['DENSITY0']['CI'] = [9337.9885256493308, 23130.782791203092, 7697.2821104556024]
% G43.16
% [Param #0     DENSITY0 =      4.30989 +/-        0.108066   Range:     [1,8],
%  Param #1      COLUMN0 =      12.1953 +/-       0.0535173   Range:   [11,16],
%  Param #2   ORTHOPARA0 =     -1.25075 +/-         1.31576   Range:[-3,0.477121],
%  Param #3 TEMPERATURE0 =       28.037 +/-         19.9428   Range:    [5,55],
%  Param #4      CENTER0 =      40.3406 +/-       0.0102343   Range:   [35,45],
%  Param #5       WIDTH0 =     0.765835 +/-       0.0100109   Range:   [0,inf)]
% stats_dict['DENSITY0']['CI'] = [10105.478740355829, 20412.420448321209, 11646.197755316312]

% codes/make_specfigure.py
\FigureTwo
          {figures/G43.17+0.01_40kmscloud_MCMCfit_nolegend_MCMCoverplot.pdf}
          {figures/G43.16-0.03_40kmscloud_MCMCfit_nolegend_MCMCoverplot.pdf}
{Optical depth spectra of the \oneone and \twotwo lines toward the two W49
lines of sight, \north (left) and \south (right).   The grey bars show the
1-$\sigma$ error bars on each data point.  The red lines show 100 realizations
from an MCMC fit of the \ortho \oneone and \twotwo lines using the LVG model
grid.  The blue lines show the hyperfine components that make up the \oneone
and \twotwo lines for the optimal fit; the \oneone line is resolved into two
components in the \north spectrum.  The residuals of the fit are shown offset
below the spectra with the residuals of the above 100 MCMC realizations
overplotted in red.  The optical depth ratio falls in a regime where
temperature has very little effect on the observed depth and there is no
degeneracy between low and high densities \citep{Ginsburg2011a}.  }
{fig:h2codensg43}{1}

% fitted using ~/work/h2co/G43.17+0.01/fit_small_lines.py, specifically the MC40 million-long chains
% and G43.16-0.03/fit_small_lines.py
% \todo{Revise or remove this - fitting done to each line independently, focused
% on integral of line}
% We performed line fits to both lines simultaneously using a Markov-chain
% monte-carlo approach, assuming uniform priors across the modeled parameter
% space and independent gaussian errors on each spectral bin.  The density
% measurements are very precise, with $log n = 4.3^{+0.16}_{-0.48} \percc$
% (95\% confidence interval) and $n=4.2^{+0.24}{-0.64}$ \percc for
% G43.17+0.01 and G43.16-0.03 respectively.  The lower bounds correspond to high ortho-to-para
% ratios OPR>1.  While these are fairly precise measurements of gas density, we
% now need to examine exactly what gas we have measured the density of.

We performed fits of the optical depth spectra to each line independently. In
both objects, we found that the centroids and widths agreed (see Table
\ref{tab:obs}).  The ratio of the line integrals is used for further analysis
%At this level of precision, the 
%density measurements are dominated by systematic uncertainties in temperature and
%the ortho-to-para ratio of \hh.  
%However...
% and collision rate uncertainties - which limit the accuracy to $\sim50\%$ using
% the \citet{Green1991} rates
% \citep{Zeiger2010}.  

\Table{lcc}
{Fitted Parameters}
{  & \oneone & \twotwo}
{tab:obs}
{
\north    &                          &                         \\
\hline
Centroid  & $  39.54^{+   0.01}_{-   0.01}$  & $  39.55^{+   0.06}_{-   0.06}$ \\
Width     & $   0.37^{+   0.01}_{-   0.02}$  & $   0.45^{+   0.07}_{-   0.08}$ \\
Peak      & $  0.114^{+  0.004}_{-  0.004}$  & $  0.015^{+  0.002}_{-  0.002}$ \\
Integral  & $  0.107^{+  0.002}_{-  0.002}$  & $  0.016^{+  0.002}_{-  0.002}$ \\
Ratio     & $   6.49^{+   0.84}_{-   0.67}$  & \\
\hline
\south    &                            &                         \\
\hline
Centroid  & $  40.35^{+   0.04}_{-   0.03}$  & $  40.36^{+   0.23}_{-   0.22}$ \\
Width     & $   0.72^{+   0.04}_{-   0.04}$  & $   0.84^{+   0.23}_{-   0.31}$ \\
Peak      & $  0.071^{+  0.003}_{-  0.003}$  & $  0.008^{+  0.002}_{-  0.002}$ \\
Integral  & $  0.130^{+  0.005}_{-  0.005}$  & $  0.018^{+  0.004}_{-  0.004}$ \\
Ratio     & $   7.32^{+   2.31}_{-   1.43}$  & \\
}{
Centroid and width are in \kms, peak is unitless (optical
depth), and the integral is in optical depth times \kms.  The errors represent
95\% credible intervals (2-$\sigma$).}


% The measured density is much higher than the \thirteenco-measured cloud-average
% density $n\approx 400$ \percc \citep[for cloud
% GRSMC\_G043.04-00.11;][]{Roman-Duval2010a}, with
% $n_{\formaldehyde}/n_{\thirteenco} \approx 50$.  The discrepancy is worse using
% the \citet{Simon2001a} cloud-averaged density $n\approx 100$ \percc.

% Our density measurements are about 4$\times$ higher than CO/CI LVG density
% measurements from \citet{Plume2004a}, though those measurements rely on
% uncertain abundances and are fairly sensitive to temperature.

Since the W49 lines of sight are clearly on the outskirts of the foreground
cloud, not through its center, it is unlikely that these lines of sight
correspond to a centrally condensed density peak (e.g., a core).  The
comparable line ratios observed through two different lines of sight separated
by $\sim 2$ pc supports this claim, since a core would result in a much higher
\twotwo optical depth.


% Using
% Figure 4 of \citet{Ginsburg2011a}, we can `turbulence-correct' the density
% measurements, but even in the most extreme case with a turbulent density
% distribution lognormal width $\sigma_s = 1.5$, the correction is only a factor
% of 2.5, reducing the discrepancy to a factor of $\sim20$.

% We should then ask, if there is gas at high density, how much is at this density?
% To address this question, we'll assume that the densities in all of the \formaldehyde
% lines of sight in the cloud are the same, and compare the \thirteenco and
% \formaldehyde derived column densities.  The \oneone line robustly reflects the
% total \formaldehyde column, even though it does not constrain the density
% without a corresponding \twotwo detection.

% Comparing the integrated \formaldehyde lines to the integrated \thirteenco
% lines, the integrated \formaldehyde column densities are
% $N_{\ortho} = 2.03\ee{12} $ and $1.56\ee{12}$ \persc for G43.16
% and G43.17 respectively.
% The \thirteenco integrated spectra have brightness $T_{MB} = 2.6$ K and $1.3$ K
% for G43.16 and G43.17 respectively.  Using the cloud-averaged excitation
% temperature for this cloud, $\tau_{13}=2.3$ and $0.6$ respectively, so
% \citet{Roman-Duval2010a} equation 3 yields column densities $N_{13} = 6.2\ee{15}
% $ and $1.6\ee{15}$ \percc respectively.  Assuming a \thirteenco abundance relative to \hh,
% $X_{13} = 1.8\ee{-6}$ \citep[consistent with ][]{Roman-Duval2010a}, the
% resulting \hh column densities are 3.5\ee{21} and 9.0 \ee{20} \percc
% respectively.  The abundances of \ortho relative to \thirteenco are 3.2\ee{-4}
% and 9.8\ee{-4} respectively, or relative to \hh, 5.8\ee{-10} and 1.7\ee{-9},
% which are entirely consistent with other measurements of $X_{\ortho}$.  
%These
%are relatively modest column densities, with $A_V=17$ and 4.5;
%these measurements are consistent with \citet{Plume2004a} if the different
%A_V/N(H_2) conversions are corrected.

% These measurements for a specific cloud validate the statistical argument made
% in \citet{Ginsburg2011a}.  

% However, upon closer inspection of the cloud
% morphology, the real explanation may be simple: the filling factor of gas
% within the GMC is small on large scales, not local scales.  The implied volume
% filling factor from this analysis and the \citet{Ginsburg2011a} analysis is
% $\sim10^{-2}$; the assumption of spherical symmetry is therefore extremely
% poor.  

% This low filling factor has major implications for the gas: if it is in
% gravitational collapse, the free-fall times are shorter by an order of
% magnitude than usually assumed.  The long lifetimes of GMCs therefore implies
% that the cloud cannot be undergoing gravitational collapse, but instead
% maintains a turbulent equilibrium.  \todo{Strengthen this argument...}
% 
% It also demonstrates that density-based star-formation thresholds do not
% independently predict star formation \citep{Parmentier2011a}.  Star formation
% cannot simply be driven by the free-fall time of gas, since apparently much of
% the gas above $n>10^4$ \percc is not in free-fall.

% 3c111 is in california, not 3c123
% \subsection{Comparison to 3C123 and the California Nebula}
% The radio source 3C123, an active galactic nucleus, is often used as a flux
% calibrator for radio telescopes.  We used it for that purpose in our GBT
% observations, and detected the \twotwo line.  \citet{Liszt1995a} detected the
% \oneone line with the NRAO 43m telescope.  The line ratio in front of 3C123 is
% approximately $\tau_{1-1}/\tau_{2-2} \approx 10$, which indicates a density
% $n\approx10^{3.6}$ \percc.  This density is significantly lower than in the
% W49 40 \kms cloud, but still higher than expected in an inactive GMC
% \citep[which this is][]{Harvey2013a}.
%
% 3c111 may have VLBA, VLA observations


% A low filling-factor may have major impact on analyses of the distribution
% functions of column density that have recently become popular
% \citep[e.g][]{Kainulainen2009}.

\section{Turbulence and \formaldehyde}
Supersonic interstellar turbulence can be characterized by its driving mode,
Mach number $\mathcal{M}$, and magnetic field strength. 
We start by assuming the gas density follows a lognormal distribution, defined
as 
\begin{equation}
    \label{eqn:lognormal}
    P_V(s) = \frac{1}{\sqrt{2 \pi \sigma_s^2}} \exp\left[-\frac{(s+\sigma_s^2/2)^2}{2 \sigma_s^2}\right]
\end{equation}
\citep{Padoan2011b,Molina2012a}
where the subscript $V$ indicates that this is a volumetric density
distribution function.  The parameter $s$ is the logarithmic
density contrast, $s\equiv\ln(\rho/\rho_0)$.
The width of the turbulent density distribution
is given by
\begin{equation}
    \label{eqn:sigmas}
    \sigma_s^2 = \ln\left(1+b^2 \mathcal{M}^2 \frac{\beta}{\beta+1}\right)
\end{equation}
where $\beta= 2 c_s^2/v_A^2 = 2 \mathcal{M}_A^2/\mathcal{M}^2$ and $b$ ranges
from $b\sim1/3$ (solenoidal, divergence-free forcing) to $b\sim1$ (compressive, curl-free)
forcing \citep{Federrath2010a}.  


The observed \formaldehyde line ratio roughly depends on the \emph{mass-weighted}
probability distribution function (as opposed to the volume-weighted
distribution function, which is typically reported in simulations).  We first
examine the implications of assuming a lognormal distribution for the
mass-weighted density.
% such that $p_m(s) = \rho \cdot p_s(s)$, or
% \begin{equation}
%     \label{eqn:lognormal}
%     p_m(s) = \frac{s}{\sqrt{2 \pi \sigma_s^2}} \exp{\left(-\frac{(\ln(\rho/\rho_0))^2}{2 \sigma_s^2}\right)}
% \end{equation}
% where we have assumed a lognormal form for $p_m(s)$.  
%Other forms of the density PDF will be addressed in Section \ref{sec:simpdfs}.

We use large velocity gradient (LVG) models of the \formaldehyde lines, which are computed assuming a
fixed local density, as a starting point to model the observations of
\formaldehyde in turbulence.   Starting with a fixed \emph{volume-averaged}
density $\rho_0$, we compute the observed \formaldehyde optical depth in both
the \oneone and \twotwo
line by averaging over the mass-weighted density distribution, $P_M\equiv\rho P_V$.
\begin{eqnarray}
    \label{eqn:tauintegral}
    \tau_{\nu}(\rho_0) &=& \int_{0}^\infty \frac{\tau_{\nu,p}(\rho)}{N_p} P_M(\ln \rho/\rho_0) d \ln \rho\\
                       &=& \int_{0}^\infty \frac{\tau_{\nu,p}(\rho_0 e^s)}{N_p} P_M(s) d s
\end{eqnarray} % there's too much spoo in my fleem http://dilbert.com/strips/comic/1997-04-27/
$\tau_p(\rho)/N_p$ is the optical depth \emph{per particle} at a given density, where $N_p$ is the column
density (\perkmspc) from the LVG model.
We assume a fixed abundance of \ortho relative to \hh
(i.e., the \formaldehyde perfectly traces the \hh).\footnote{While there is
building evidence that there is \hh not traced by CO
\citep{Shetty2011b,Shetty2011a}, the \formaldehyde and CO should be tracing the
same gas, as \formaldehyde abundances have typically been observed to be
consistent with CO abundances.  \formaldehyde deficiency is also most likely to
occur on the outskirts of clouds where the total gas density is expected to be
lower, so our measurements should be largely unaffected by abundance variation
within the cloud.}

Figure \ref{fig:lvgsmooth}
shows the result of this integral for an abundance of \ortho relative to \hh, 
$X(\ortho)=10^{-8.5}$, where the X-axis shows $\rho_0 = \rho(\hh)$ and the Y-axis
shows the observable optical depth ratio of the two \formaldehyde centimeter
lines.
The LVG model, which assumes a single density
(or a Dirac $\delta$ function as the density distribution), is shown along with
`smoothed' versions of the model which take into account realistic turbulent
gas distributions.  Because the \formaldehyde \twotwo line requires a higher
density to be ``refrigerated'' into absorption, any spread of the density
distribution effectively increases the \twotwo line without decreasing the
\oneone line and therefore decreases the \oneone/\twotwo ratio.  

% Abundance limits: 
% At n=24553, logc=12.38 (best fit)
% tex=5.66
% tau=1.92
% N = 2.6e14 integral(tex * tau / (1-exp(-5.3/tex)))
%   = 2.6e14 * 1.92 ... no
% tau_13 = -log(1-0.189 T_A / ((exp(5.3/tex)-1)^-1-0.16))
%        = 
%
%
% ... "measuring" abundance yields X=-8.26
% EXCEPT THAT'S WRONG!!! I forgot to integrate the line...
% ugh, double wrong... really more like -8.5


%% pp removed: we don't really need to worry about the effect on the column density
%% for the theoretically computed plot since we have constraints on the column...
%which necessarily implies a higher
%column density of \ortho for the higher densities in Equation
%\ref{eqn:tauintegral}.  As long as the \formaldehyde lines are optically thin,
%this approach should yield the right \emph{ratio} of the two lines, although the
%absolute optical depths may be substantially smaller because of lower total
%\ortho columns.  An example of this smoothing is shown in Figure
%\ref{fig:lvgsmooth}.

% /Users/adam/work/h2co/pilot/plotcodes/lognormal_density_massweighted.py
% path: /Volumes/disk5/Users/adam/work/h2co/pilot/figures/models/lognormalsmooth_density_ratio_massweight_logopr0.0_abund-9.png
% cp ~/work/h2co/pilot/figures/models/lognormalsmooth_density_ratio_massweight_withhopkins_logopr0.0_abund-9.png figures/
% cp ~/work/h2co/pilot/figures/models/lognormalsmooth_density_ratio_massweight_withhopkins_logopr0.0_abund-9_withG43.png figures/
\FigureThreePDF
    {figures/lognormalsmooth_density_ratio_massweight_withhopkins_logopr0.0_abund-8.5_withG43}
    {figures/lognormalsmooth_density_tau_oneone_massweight_withhopkins_logopr0.0_abund-8.5_withG43}
    {figures/lognormalsmooth_density_tau_twotwo_massweight_withhopkins_logopr0.0_abund-8.5_withG43}
{The predicted \formaldehyde \oneone/\twotwo ratio and optical depths as a function of volume-weighted mean
density for a fixed abundance relative to \hh $X(\ortho) = 10^{-8.5}$  with \hh
ortho/para ratio 1.0.  The legend shows the effect of smoothing with different
mass distributions. % as described in Equation \ref{eqn:sigmas}.  % and \ref{eqn:lognormal}.  
The solid line shows the predicted values
with no smoothing (i.e., a $\delta$-function density distribution).
The blue point shows the G43.17 measurement.  The horizontal red error bars
show the limits on the mean volume density, $\rho_V$, and the vertical blue error
bars show the 95\% credible interval for the \formaldehyde line measurements.
}
{fig:lvgsmooth}{3.5in}

\subsection{The $\rho$-PDF in \GRSMC}
\label{sec:grsmcturb}
We use the density measurements in \GRSMC to infer properties of that
cloud's density distribution.  The observed ratio for the \north sightline in
GRSMC 43.30-0.33 is shown in Figure \ref{fig:lvgsmooth} as a blue point.  
%Very wide distributions are required to match the observations.

% We measure the abundances of \ortho relative to \thirteenco using the LVG
% models for \formaldehyde and the \citet{RomanDuval2010a} optical-depth
% corrections for \thirteenco.
% \todo{ The measured values are $X(\ortho/\thirteenco) = 3.2\ee{-4}$ and 3.5\ee{-4} for G43.16 and G43.17
% respectively, or relative to \hh, $X(\ortho/\hh)=5.8\ee{-10}$ and 6.2\ee{-10} }
% \todo{The above are incorrect, the latest are: \\ 
% G43.16: Abundance relative to H2:    -8.46394793495 \\ 
% G43.17: Abundance relative to H2:    -8.52349945355 \\
% }
% \todo{TODO: Correct G43.16 numbers; G43.17 is accurate}, which are
% entirely consistent with other measurements of $X_{\ortho}$
% \citep{Johnstone2003a} and allow us to use constant abundance LVG models for this
% analysis\footnote{Higher abundances of \formaldehyde have rarely been observed,
% but lower abundances are common in cores.  The effect of lower abundance is to
% \emph{increase} the inferred $\sigma_s$ in the analysis below, so our
% assumption of $X\lesssim5\times10^{-9}$ is conservative}.  
% why was this commented out? it's reported below TAG:MEASURE
% The observed
% formaldehyde line ratio $\tau_{1-1}/\tau_{2-2} = 6.99^{+0.72}_{-0.56} $ (95\%
% CI), while the volume averaged mean density of the cloud 8 \percc $\lesssim
% \rho_0 < 150$ \percc.


\Table{ccccc}
{Fitted Distribution Parameters}
{Parameter & Lognormal & Hopkins}
{tab:datafits}
{
$b$&-&-&-&0.64 \\
$\sigma_s$&2.9&$^{2.2}_{3.7}$&2.4&$^{2.0}_{3.5}$ \\
$\sigma_s | M$&-&-&2.3&$^{1.9}_{2.9}$ \\
$T$&-&-&0.25&$^{0.19}_{0.34}$ \\

}
{\\ The 95\% credible intervals are reported showing the lower and upper limits
of the bounded regions.  For the $b$ parameter, only the lower limit is shown.
$\sigma_s|M$ is the measured width including the measured Mach number range.
}


% we use
% $\sigma_s=2.5$ as the ``best fit'' distribution, but note that any value in the
% range $1.5<\sigma_s<2.2$ \todo{TODO: CHECK PRECISE VALUES} is permissible for the
% lognormal distribution or $2.3 < \sigma_s < 3$ for the Hopkins distribution.

% sigma^2 = (1.5-2.2) = ln ( 1 + b^2 M^2 ) 
% TODO: The 3D mach number allows for MUCH higher sigma and requires LOW b;
% this is bad for my hypothesis.  That misinterpretation could cost a lot...

% \todo{Below this point, calculations are done assuming the 1D Mach number 5.1-9.1;
% if the 3D is used in its place, the $b$ value is lower, $\sim1/3 - 0.5$.
% However, if $\sigma$ is used in place of the FWHM, the 3D Mach range is
% $3.7-6.6$ and $b$ is higher.}

% TAG:MEASURE
Assuming the thermal dominates the magnetic pressure ($\beta>>1$), we can fit
$\sigma_s$ from the measurement in Figure \ref{fig:lvgsmooth}.  Using two
different forms for the density distribution, and using only the $\tau$
measurements as a constraint, we derive the value of $\sigma_s$ in Table
\ref{tab:datafits}.
% assuming a lognormal
% distribution, the best-fit
% $\sigma_s=2.8$, with a 95\% confidence interval $2.36 < \sigma_s < 3.49$.

It is possible to provide direct constraints on the $b$ value by using information about
the Mach number from line-of-sight velocity dispersion measurements.  Assuming
a temperature $T=10$ K, consistent with both the \formaldehyde and CO
observations \citep{Plume2004a}, the sound speed in molecular gas is $c_s=0.19$
\kms.  The gas is unlikely to be much colder, so using this sound speed will
provide at least a lower limit on the Mach number.  The observed line FWHM in
G43.17 is 0.95 \kms for \formaldehyde and 1.7 \kms for \thirteenco 1-0,
% \citep[although warm gas tracers have a larger linewidth, FWHM(H$_2$$^{18}$O)
% and FWHM(C I) = 3.2
% \kms;][]{Plume2004a},
so the 3-D Mach number of the turbulence is
$$\mathcal{M}_{3D} \equiv 3^{1/2} \mathcal{M}_{1D} \approx \frac{3^{1/3}}{(8\ln 2)^{1/2}} FWHM / c_s $$
or $\mathcal{M}_{3D} = 3.8 - 6.6 $, ranging from the \formaldehyde to the \thirteenco width.

\Figure{figures/mach_vs_b_lognormal_mcmc_contours_abundance-8.5.pdf}
{Contours of the MCMC fit to the \formaldehyde optical depths with the cloud
mean density restricted to $10~\percc < \rho_V < 200~\percc$.  Using the constraints on the Mach
number based on the \north line of sight, we can infer $b>0.6$.
\todo{Add Mach vs $\sigma_s$.}\\
\todo{Add Mach vs $b$ \emph{with} constraints - it makes the situation much more clear}.
}{fig:lognormalmcmc}{0.5}{0}

\todo{Constraints on $b$}
% In principle, we should be able to then measure the compressibility coefficient
% $b$ using our constraints on $\mathcal{M}$ and Equation \ref{eqn:sigmas}.
% However, even if we select the low end of the $\sigma_s$ values $\sigma_s=2.2$
% and the highest Mach number $\mathcal{M} = 6.6$, the $b$ values are
% unrealistic: we have measured a lower-limit $b>1.7$.  This measurement indicates
% that the lognormal density distribution is inadequate to describe the cloud.
% Since there is no allowed parameter space, the fitted parameters are left
% blank in Table \ref{tab:datafits}.
% % the allowed values of $\sigma_s$ given the line-width based limits on
% % $\mathcal{M}$ range from 1.8-2.1 for $b=1$ and 1.2-1.5 for $b=1/3$.  If
% % magnetic pressure is significant, the allowed values of $\sigma_s$ drop.

% Given that the observed mean cloud density is $n(\hh)\lesssim10^2 \percc$, Figure
% \ref{fig:lvgsmooth} shows that only the most extreme values of $\sigma_s$ can
% explain the mean density.  Even if the cloud is extremely oblate, e.g. with a
% line-of-sight axis $0.1\times$ the plane-of-sky axes, $\sigma_s > 1.5$ is
% required.

% This PP is OK
As one possible alternative, we use the
\citet{Hopkins2013a} density distribution with $T-\sigma_s$ and $T-\mathcal{M}_C$
relations fitted to measurements from a series of simulations
\citep{Kowal2007a,Kritsuk2007a,Schmidt2009a,Federrath2010a,Federrath2012a,Konstandin2012a,Molina2012a}.
We derive a $T$ value from 
\begin{equation} 
    \label{eqn:Tsigma}
    T(\sigma_s) = 0.25 \ln (1+0.25 \sigma_s^4 \left(1+T(\sigma_s)\right)^{-6})
\end{equation}
where $T$ is an ``intermittency'' parameter that indicates the
deviation of the distribution from lognormal\footnote{Equation \ref{eqn:Tsigma}
is a transcendental equation, so we use root-finding to determine $T$.}.  
The intermittency parameter is described in \citet{Hopkins2013a} as a unitless
parameter which increases with Mach number and is correlated with the strength
of the deviations from the mean turbulent properties as a function of time.
Its physical meaning beyond these simple correlations is as yet poorly understood.

Using the \citet{Hopkins2013a} distribution, we find \todo{$\sigma_s=2.4$}, with
a 95\% credible interval \todo{$2.0 < \sigma_s < 3.5$}.  These values are
compatible with the observed Mach numbers.  Using the relation 
\begin{equation}
    \label{eqn:McMT}
    b \mathcal{M} = \mathcal{M}_c  \approx 20 T
\end{equation}
from \citet{Hopkins2013a} Figure 3, we can derive a lower limit \todo{$b>0.64$} at the
95\% credible level.  However, there is intrinsic uncertainty in the
coefficient in Equation \ref{eqn:McMT} that we have not accounted for.  
% ($b>0.49$ at 99.9\% confidence).


% Using the $\sigma_s=2.5$ distribution, which is
% just barely consistent with the observations, $T=0.29$, and based on
% \citet{Hopkins2013a} Figure 3, the compressive Mach number $\mathcal{M}_c ~ 20
% T \approx 5.8$.   Compared to the Mach number restrictions from the line width,
% this $\mathcal{M}_C$ implies a compressive-to-total ratio $b > 0.6$.
% In [74]: hopkins_pdf.T_of_sigma(2.5)
% Out[74]: 0.2906836447265763
% 
% In [75]: hopkins_pdf.T_of_sigma(2.5) * 20
% Out[75]: 5.813672894531527
% 
% In [76]: hopkins_pdf.T_of_sigma(2.0)
% Out[76]: 0.22018342653110817
% 
% In [77]: hopkins_pdf.T_of_sigma(2.0) * 20
% Out[77]: 4.403668530622164

The restrictions on $\sigma_s$ using either assumed density distribution are
strong indications that compressive forcing must be a significant, if not
dominant, mode in this molecular cloud.  
% Since magnetic fields have the
% opposite effect of compressive turbulence on the density distribution, magnetic
% fields cannot explain the observations.
% If magnetic fields are in balance with
% or dominate thermal pressure in this cloud, $\beta\gtrsim2/3$, the forcing must
% be predominantly compressive, with $b>1/3$.
% Crutcher & others seem not to have detected Zeeman splitting in this cloud
All of the systematic uncertainties tend to require a \emph{greater} $b$
value, while we have already inferred a lower-limit
that is higher than others have observed \citep{Brunt2010c,Kainulainen2013a}.
Temperatures in GMCs are typically 10-20 K, and we assumed 10 K: warmer
temperatures increase the sound speed and therefore decrease the Mach number. If
the cloud is warmer, the $b$ values again must be higher to account for the measured
$\sigma_s$.  Magnetic fields similarly have the inverse effect of $b$ on
$\sigma_s$, with decreasing $\beta$ requiring higher $b$ for the same
$\sigma_s$.
Lower abundance shifts all curves in Figure
\ref{fig:lvgsmooth} up and to the right, again decreasing $\sigma_s$.

\Figure{figures/lognormalsmooth_density_distributions_sigma2.0.png}
{ %\todo{CUT in final version; this is just a demonstration of the different distributions.} 
Example volume- and mass-weighted density distributions with
$\sigma_s=2.0$.  The vertical dashed lines show $\rho = 15$ and $\rho=10^4$,
approximately corresponding to the volume-averaged mean density of \GRSMC
and the \formaldehyde-derived density.}
{fig:distributions}{0.5}{0}

% For G43.16-0.03:
% {'std': 0.11326554809612598, 'med': 4.3331511700465288, 'CI': [9368.5009376639355, 21535.312095032383, 11915.556844413139], 'quantiles': {2.5: 4.08517676736491, 97.5: 4.5244074037392616, 75: 4.4038773064784564, 84.134474606854297: 4.4358245264528868, 50: 4.3331511700465288, 15.865525393145708: 4.2122132860734167, 2.2750131948179195: 4.078075142949765, 25: 4.2563141956721111, 97.724986805182084: 4.5273461301744797}, 'mad': 0.10889113037076946, 'mean': 4.3251999406282646, 'logCI': [0.24797440268161886, 4.3331511700465288, 0.19125623369273281]}
% 

% \subsection{Simulated PDFs}
% \label{sec:simpdfs}
% Real turbulent PDFs are not truly lognormal, though often they are
% well-approximated as lognormals.  We have used some of the PDFs from
% \citet{Federrath2012a} to perform additional smoothing and determine
% whether deviations from lognormal can explain the observed density contrasts.
% 
% To perform the smoothing, we converted the simulation's volume-weighted PDF to
% a mass-weighted PDF using Equation \ref{eqn:lognormal} and used an identical
% PDF shape for each mean density (i.e., we kept the shape of the PDF the same
% but changed its mean for use in Equation \ref{eqn:tauintegral}).  Results of this process
% are shown in Figure \ref{fig:rescalepdfs}.

% ~/work/h2co/simulations/federrath_pdfs.py
% cp ~/work/h2co/simulations/VolumeVsMassWeighting.png figures/
%\Figure{figures/VolumeVsMassWeighting.png}
%{Mass-weighted mean density vs volume-weighted mean density for a variety of
%turbulent distributions.  The black dashed line shows the $\rho_M = \rho_V$
%relation.  The other lines show the relationship between $\rho_M$ and $\rho_V$
%for different lognormal widths $\sigma_s = \sigma_{\ln \rho}$ and values of $T$
%in the \citet{Hopkins2013a} distribution.  The measured densities for the cloud
%GRSMC 43.30-0.33 are shown with conservative error bars; the vertical bars show
%the 95\% confidence interval for the \formaldehyde density, while the
%horizontal bars show the range $8-150$ \percc, the full range of geometrically
%allowed volume-averaged densities.  However, the \formaldehyde density measurement
%is not strictly a mass-weighted density (see Equation \ref{eqn:tauintegral}),
%so the positions of the data in this figure are somewhat misleading.
%}{fig:volvsmass}{0.5}{0}


\section{Conclusions}
We demonstrate the use of a novel method of inferring the shape of the density
probability distribution in a molecular cloud using \formaldehyde densitometry
in conjunction with \thirteenco-based estimates of total cloud mass.

Our data show evidence for compressively driven turbulence in a
non-star-forming giant molecular cloud.  Such high compression in a fairly
typical GMC indicates that compressive driving is probably a common feature of
all molecular clouds.

Our data also indicate that a lognormal distribution is inadequate to describe
the density PDF.  Instead, an intermittent distribution such as that presented
in \citet{Hopkins2013a} is an acceptable fit to the data.

% \FigureTwoAA{figures/federrath_pdfs_volume_mach10.png}{figures/federrath_pdfs_recentered_massweighted_fitted_mach10.png}
% {PDFs from \citet{Federrath2012a}.  (a) Volume-weighted PDFs for various
% simulations with $\mathcal{M}=10$.  (b) Mass-weighted PDFs from the same
% simulations as (a).  These PDFs have been recentered such that they
% have a mean overdensity $s=0$.
% }{fig:rescalepdfs}{1}{5in}
% 
% In order to simplify the application of these PDFs to the LVG models, we fit
% the asymmetric distributions with the sum of two lognormals with different
% means.  This approach allows for an easier exploration of parameter space.
% An example demonstrating that two lognormals is a good approximation of one of the
% compressive simulations is shown in Figure \ref{fig:fittedpdf}.
% 
% \Figure{figures/federrath_mach10_rescaled_massweighted_fitted.png}
% {The Mach 10 compressive simulation PDF from \citet{Federrath2012a} is shown in
% blue with the best-fit single lognormal in green and sum of two lognormals in
% red.  The two-lognormal approximation is a good fit to the simulated PDF.}
% {fig:fittedpdf}{0.5}{0}
% 
% To use these fitted two-lognormal distributions, we create new PDFs consisting
% of lognormals with the sample amplitude \& width ratios and the same mean
% differences as the fit in Figure \ref{fig:fittedpdf}, but with the total width
% scaled.  In Figure \ref{fig:compsmooth} [not included; see below], the reported widths for the
% ``compressive'' distributions are the widths of the wider, lower distribution
% in \ref{fig:fittedpdf}.
% 
% Upon further inspection, this approximation actually does a poor job as it
% fails to reproduce the tails, which are more important than the peak.

{\it Facilities:} \facility{GBT}, \facility{Arecibo}, \facility{VLA},
\facility{FCRAO}, \facility{CSO}

\ifstandalone
\bibliographystyle{apj_w_etal}	% or "siam", or "alpha", or "abbrv"
\bibliography{thesis}		% bib database file refs.bib
\fi


\section{Data Fitting: Summary of the Bayesian approach}
In Section \ref{sec:grsmcturb}, we performed fits to the data using a
Markov-Chain Monte-Carlo fitting approach written in \texttt{pymc}.  The spectral line
data were fit using standard approaches implemented in \texttt{pyspeckit}
(\url{pyspeckit.bitbucket.org}).  The \texttt{pyspeckit} fits yielded an
estimate of the optical depth ratio $R = \tau_{\oneone}/\tau_{\twotwo}$ with approximately
Gaussian error bars, so the ratio $R$ was implemented as a normally distributed
random variable.

The Mach number determinations from the \formaldehyde and CO lines disagree by
a factor of nearly two, so we treated the Mach number as a normally distributed
variable with $\mu=5.1$ and $\sigma=1.4$ such that an intermediate value
between the CO ($\mathcal{M}_{3d} = 6.6$) and \formaldehyde ($\mathcal{M}_{3d}
= 3.7$) measurement is weakly favored, but both measured values are allowed.
The uncertainty in the Mach number also reflects the general difficulty in determining
a Mach number from spectral line data, as noted in \citet{Schneider2013a}.

In the interest of reproducibility, the code for these models is publicly
released at \url{https://bitbucket.org/keflavich/h2co-turbulence/overview}.
\todo{The repository is currently private but if you would like access to the
source code just send me your e-mail address and I'll open it to you}.

% \section{Abundance}
% \todo{This is a summary of work in progress; it is not intended to be kept in
% the submitted document in this form.}
% In the context of the LVG model, the largest remaining uncertainty is the
% abundance.  We can take a few different approaches to minimize the effect of
% this uncertainty.
% 
% \begin{enumerate}
%  \item we can assess what measurements come out of using different abundances
%  \item we can attempt to measure the abundance of H2CO relative to 13CO and use
%      that as an input to the LVG models
% \end{enumerate}
% 
% I've begun both efforts.  Using the least-squares fit of the formaldehyde model,
% the \ortho abundance relative to \thirteenco is -2.5, or relative to \hh, -8.26.
% This analysis is done using eqn. 2 and 3 from \citet{Roman-Duval2010a} by
% assuming $T_{ex}=const$ at the value reported for GRSMC 43.30-0.33.
% 
% The important question is then - what are the effects of higher abundance?
% Unfortunately, it appears that higher abundance shifts the curves a bit down
% and left in Figure \ref{fig:lvgsmooth},  implying that lower $\sigma_s$ values
% are possible, so perhaps it will be possible to accommodate lognormal.

\section{Assumptions, caveats, and uncertainties}
There remain uncertainties that we have not explicitly explored in this paper,
which we describe and estimate here.

The precise density measurements presented here are based off of large velocity
gradient approximations \citep{Sobolev1957a} for the escape probability of line
radiation from the cloud.  This method is widely used but remains an
approximation.  In the case of \formaldehyde, it has been tested with a variety
of codes \citep{van-der-Tak2007a,Henkel1983a} but is subject to uncertainties in
the velocity gradient and system geometry.  However, in the case of the
observations in this paper, the lines were observed in the optically thin regime, and the
LVG approximation should not affect our results.

The collision rates of \formaldehyde with p-\hh, o-\hh, and He are estimated
based on computer simulations of the particles.  \citet{Troscompt2009a}
improved upon the measurements of \citet{Green1991a}, bringing the typical
collision rate uncertainty down from $\sim50\%$ in the He-based approximation
to $\sim10\%$ using full models of ortho and para \hh.  \citet{Zeiger2010a}
reported the results of using modified collision rates and noted that the
errors in the \hh density were, in the worst case, $<0.3$ dex.  With the
improved \citet{Troscompt2009a} collision rates, the uncertainties can no
longer be dominated by collision rate uncertainties.

Abundance remains a serious concern, as most studies of \formaldehyde abundance
do not observe multiple transitions and therefore do not constrain the relative
level populations.  Few systematic studies have been undertaken
\citep[e.g.][]{Liszt2006a}.  There are also general difficulties in measuring absolute
abundance of molecules, as the absolute column of \hh is rarely known to any
precision.  Most abundance measurements tend to be below $X_{\formaldehyde}<10^{-9}$
\citep{Pauls1996a,Dickens1999a,Turner1989a,Turner1993a}, which validates our
measurements, but deeper studies of abundance are warranted, particularly
studies of how \formaldehyde varies relative to CO in clouds.

The ortho-to-para ratio of \hh is a significant uncertainty in the models,
since para-\hh is more effective at ``refrigerating'' the \formaldehyde
molecules.  Values of the ortho-to-para ratio $>1$ favor lower densities by
$\sim0.3$ dex in our analysis, but we have used these lower densities in our
analysis, and therefore our results are conservative.  However, if the
ortho-to-para ratio is in reality close to zero, the density PDF must be wider.

The ``covering factor'' of foreground clouds in front of background
illumination sources is, in general, a major concern when performing absorption
measurements.  For the clouds presented in this work, the absorbing region
should be much larger than the background, as evidenced by the two
lines-of-sight with similar optical depth ratios.  However, for more detailed
studies of density variations, EVLA observations can and should be employed.

The single largest uncertainty is related to the mean properties of the GMC.
While we have accounted for these uncertainties by adopting a very conservative
range of values for the mean density (covering two orders of magnitude), it is
not entirely clear how the mean density of the cloud should be computed for
comparison to simulations and the analytic distributions.  Since this is a
foreground cloud lying in front of a rich portion of the galactic plane, the
best mass estimates are very likely to come from molecular line observations,
and therefore they are unlikely to be improved unless new wide-field CO
observations are taken, e.g. with CCAT.  

To validate our cloud mean density measurements, we have performed a dendrogram
analysis \citep{Rosolowsky2008c} on the integrated \thirteenco map of the
\GRSMC cloud.  We perform the analysis both on the large-scale $r\sim20$ pc
cloud, tracking down to 10 pc scales, and then on the individual clump that is
directly in front of W49.  We show the cloud density, computed using the
assumptions stated in the text to convert \thirteenco luminosity to mass, for
three different geometrical assumptions described in the caption of Figure
\ref{fig:dendro}.  While the clump densities are potentially higher than we
assumed in the analysis, they are probably not the appropriate numbers to
compare to the simulations we have cited, which are generally simulating entire
molecular clouds and measuring the density distribution within a large box.

\Figure{figures/GRSMC_43.30_dendrograms_density.pdf}
{Results of a dendrogram analysis of the \GRSMC cloud and the
northernmost \thirteenco ``clump'' within that cloud.  The data points represent
successively higher (and therefore smaller) contours within the integrated \thirteenco
map.  The shapes represent three different methods
for extracting the volume: squares assume spherical symmetry using the
effective radius of the contour, which is proportional to the square root of
the number of pixels.  Circles do the same, but assume that the line-of-sight
radius is 2 pc (i.e., smaller than the observed plane-of-the-sky
dimensions).  The triangles show the volume of an ellipsoid using the moments of the
contoured pixels, with volume $V=4/3 \pi R_{maj}R_{min}^2$.  The black dashed lines
indicate the range of densities allowed in our fits.}
{fig:dendro}{0.5}{0}

\end{document}
