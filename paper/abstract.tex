\title{Molecular Density Measurements with \formaldehyde: Turbulence is
    Compressively Driven}
\begin{abstract}
    Molecular clouds are supersonically turbulent.  This turbulence may govern
    the form of the initial mass function and the star formation rate of the
    gas.  It is therefore essential to understand the properties of turbulence,
    in particular the probability distribution of density in turbulent clouds.

    We present \formaldehyde volume density measurements of a non-star-forming
    cloud along the line of sight towards W49A.  This method is complementary
    to measurements of turbulence via the column density distribution and
    should be applicable to any molecular cloud with detected CO.  We show that
    turbulence in this cloud must be compressively driven, with a
    compressive-to-total ratio $b = \mathcal{M}_C/\mathcal{M}>0.6$, and that a
    lognormal distribution is unlikely to describe the cloud's density.  
\end{abstract}
