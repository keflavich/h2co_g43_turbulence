\title{The shape of the density distribution in molecular clouds: Observations with
\formaldehyde}
\begin{abstract}
    Molecular clouds are supersonically turbulent.  This turbulence governs
    the initial mass function and the star formation rate.  In order to
    understand the star formation rate, it is therefore essential to understand
    the properties of turbulence, in particular the probability distribution of
    density in turbulent clouds.

    We present \formaldehyde volume density measurements of a non-star-forming
    cloud along the line of sight towards W49A. We use these measurements in
    conjunction with total mass estimates from CO to infer the shape of the
    density probability distribution function.  This method is complementary to
    measurements of turbulence via the column density distribution and should
    be applicable to any molecular cloud with detected CO.  We show that
    turbulence in this cloud is probably compressively driven, with a
    compressive-to-total Mach number ratio $b = \mathcal{M}_C/\mathcal{M}>0.6$.  
    We measure the standard deviation of the density distribution $1.5 < \sigma_s < 1.9$
    assuming that the density is lognormally distributed.  This measurement
    represents an essential input into star formation laws.

    % \todo{Include measurements of $\sigma_s$.}

    % \todo{State what models we can exclude, if any}

\end{abstract}
