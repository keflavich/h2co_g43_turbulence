\input{preface}
\begin{document}

\textbf{Math related to Optical Depth computation}

$\tau_{LVG}(\rho)$ is computed from the LVG computations.  It is the optical
depth for a given column $N_{LVG}$ of \formaldehyde molecules assuming a velocity
difference along the line of sight of 1 km/s per pc.
The optical depth per particle is therefore the LVG $\tau$ divided by the column
density.
\begin{equation}
    \tau_p(\rho) = (\tau_{LVG}(\rho)/N_{LVG})_p
\end{equation}

Holding the volume fixed and increasing in density $n(\hh)$ either implies a
corresponding increase in column density $N_{LVG}$ or in abundance $X$.

Assuming we can compute this optical depth per particle in hand as variable
$\tau_p(\rho)$, we can then compute the average optical depth per particle for
the full density distribution.
The probability distribution $P(\rho)$ is the probability of finding a given
particle at a given density, i.e. it is the \emph{mass-weighted} distribution $P_M$.

\begin{equation}
    \label{eqn:tau1}
    \tau_p(\larrow\rho\rarrow_V) \equiv \tau_p(\rho_0) = \int_0^\infty \tau_p(\rho) P_M(\rho) d\rho
\end{equation}

However, we are dealing with a lognormal distribution $P_M(s)$, so it is
convenient to recast Equation \ref{eqn:tau1} in terms of $\ln(\rho)$.
Below are three equivalent ways of writing this equation, though the first
mixes variables ($s$ depends on $\rho$: $s\equiv \ln(\rho/\rho_0)$, or $\rho = \rho_0 e^s$).

\begin{eqnarray}
    \label{eqn:tau2}
    \tau_p(\rho_0) & = & \int_{-\infty}^\infty \tau_p(\rho) P_M(s) d s \\
                 & = & \int_{-\infty}^\infty \tau_p(\rho) P_M(\ln (\rho/\rho_0)) d\ln (\rho/\rho_0) \\
                 & = & \int_{-\infty}^\infty \tau_p(\rho_0 e^s) P_M(s) d s 
\end{eqnarray}

The quantity we actually want to model, though, is the total optical depth observed,
\begin{equation}
    \label{eqn:totaltau1}
    \tau_\nu(\rho,N) = N \tau_p(\rho)
\end{equation}

As long as we operate in the optically thin regime, such that $\tau_p$ is independent of $N$,
we can use the average density on both sides of Equation \ref{eqn:totaltau1}.

\begin{equation}
    \label{eqn:totaltau2}
    \tau_\nu(\rho_0,N) = N \tau_p(\rho_0)
\end{equation}

Putting this all together, the observed optical depth $\tau_\nu(\rho_0)$ is given by

\begin{eqnarray}
    \tau_\nu(\rho_0) & = & N \int_{-\infty}^\infty \tau_p(\rho_0 e^s) P_M(s) d s \\
    \tau_\nu(\rho_0) & = & X \rho_0 \int_{-\infty}^\infty \tau_p(\rho_0 e^s) P_M(s) d s 
\end{eqnarray}

\end{document}
